\pagebreak
\section{ECCO Data Filenames and Supporting Conventions}


ECCO Version 4 Release 4 (V4r4) follows a structured filename convention to organize its extensive datasets. The filenames encode essential information about the dataset's content, grid type, temporal resolution, and version. Below is an overview of the naming convention.

\subsection{General filename Structure}
\par \vspace{0.25cm}

\par For ECCO dataset product files:

\par \vspace{0.5cm}
\begin{center}
\small{\textbf{\fontfamily{lmtt}\selectfont{[ShortName]\_[TemporalResolution]\_[Indicative Time]\_ECCO\_[Version]\_[GridType].<File Type>}}}
\end{center}
\par \vspace{0.5cm}
\par For ECCO data geometry and grid files:

\par \vspace{0.5cm}
\begin{center}
    \small{\textbf{\fontfamily{lmtt}\selectfont{[ShortName]\_ECCO\_[Version]\_[GridType].<File Type>}}}
\end{center}
% \texttt
\par \vspace{0.25cm}

\begin{center}
\begin{tabular}{m{0.18\textwidth} m{0.75\textwidth}}
    \multicolumn{1}{c}{\textbf{Keypoins}} & \multicolumn{1}{c}{\textbf{Description}} \\ \hline
    ShortName: & Describes the dataset variable (e.g., SSH for sea surface height, TEMP\_SALINITY for temperature and salinity). \\ \hline 
    TemporalResolution: & Indicates the time averaging or snapshot type (e.g., DAILY, MONTHLY, SNAP). \\ \hline
    GridType : & Specifies the grid used (e.g., \textbf{native\_llc90} for the native ECCO grid or \textbf{laton\_0p50deg} for interpolated 0.5 \textdegree lat-lon grid). It can also be a specific information such as \textbf{1D} to indication the one-dimentional dataset files over the full period of ECCO data availability.\\ \hline
    Version: &  Identifies the ECCO release version (e.g., V4r4). \\ \hline
    Indicative Time: & Encodes the time period covered by the file (e.g., monthly files use YYYY-MM, daily files use YYYY-MM-DD) \\ \hline
    File Type & netCDF or Zarr data format. \\ \hline
\end{tabular}
\end{center}

\par \vspace{0.25cm}
\subsection{Examples}
\par Daily atmosphere surface temperature, humidity, winds, and pressure on the lat-lon-cap 90 (llc90)
native model grid (first) and on the interpolated regular 0.5-degree lat-lon grid (second) from ECCO V4r4 on 2017-12-29.
\begin{itemize}
    \item ATM\_SURFACE\_TEMP\_HUM\_WIND\_PRES\_day\_mean\_2017-12-29\_ECCO\_V4r4\_native\_llc0090.nc
    \item ATM\_SURFACE\_TEMP\_HUM\_WIND\_PRES\_day\_mean\_2017-12-29\_ECCO\_V4r4\_latlon\_0p50deg.nc
\end{itemize}

\par Dynamic sea surface height interpolated on the latlon regular 0.5-degree model grid from ECCO V4r4 on 2017-12-29.

\begin{itemize}
    \item SEA\_SURFACE\_HEIGHT\_day\_mean\_2017-12-29\_ECCO\_V4r4\_native\_llc0090.nc
\end{itemize}

\par One-dimentional field of global mean atmospheric surface pressure over the ocean and sea-ice fromthe ECCO V4r4.
\begin{itemize}
    \item GLOBAL\_MEAN\_ATM\_SURFACE\_PRES\_snap\_ECCO\_V4r4\_1D.nc
\end{itemize}

\par Geometric parameters for the regular 0.5-degree lat-lon grid from ECCO V4r4.
\begin{itemize}
    \item GRID\_GEOMETRY\_ECCO\_V4r4\_latlon\_0p50deg.nc
\end{itemize}

%%% Arret ici !!!!!! Modify!!!

% \subsection{1 Overview of Filename Convention and Example Filenames}
% The filenaming convention for the GDS 2.0 is shown below. 
% \par \vspace{0.25in}

% \small{\texttt{<Dataset name>_<>_<Indicative Time>-<RDAC>-<Processing Level>\_GHRSST-<SST Type>-
% <Product String>-<Additional Segregator>-v<GDS Version>-fv<File Version>.<File Type>}}
% \par \vspace{0.25in}

% The variable components within braces (“< >”) are summarized in Table 7-1 below and detailed in the
% \textbf{should not} be used in any GHRSST code or the <Additional Segregator> element. Example
% filenames are given later in this section. While no strict limit to filename length is mandated, RDACs
% are encouraged to keep the length to less than 240 characters to increase readability and usability.

% %Table 7-1. GDS 2.0 Filenaming convention components
% \begin{longtable}{|p{0.2\textwidth}|p{0.3\textwidth}|p{0.5\textwidth}|}
% \caption{GDS 2.0 Filenaming convention components}
% \label{tab:filenaming conventions}
% \\ \hline
% \rowcolor{lightgray}
% \textbf{Name} & \textbf{Definition} & \textbf{Description} \\ \hline 
% \endfirsthead
% <Indicative Date> & YYYYMMDD & The identifying date for this data set. See Section 7.2. \\ \hline
% <Indicative Time> & HHMMSS & The identifying time for this data set. The time used is dependent on the <Processing Level> of
% the data set: L2P: start time of granule 
% \begin{itemize}
%  \item{L3U: start time of granule}
%  \item{L3C and L3S: centre time of the collation window}
%  \item{L4 and GMPE: nominal time of analysis}
% \end{itemize}
% All times should be given in UTC. See Section 7.3. \\ \hline
% <RDAC> & The RDAC where the file was created & The Regional Data Assembly Centre (RDAC)code, listed in Section 7.4. \\ \hline
% <Processing Level> & The data processing level code (L2P, L3U, L3C, L3S, or L4) & 
% The data processing level code, defined in Section 7.5. \\ \hline
% <SST Type> & The type of SST data included in the file. & 
% Conforms to the GHRSST definitions for SST, defined in Section 7.6 \\ \hline

% <Product String> &
% A character string identifying the
% SST product set. The string is
% used uniquely within an RDAC
% but may be shared across
% RDACs. & 
% The unique “name” within an RDAC of the
% product line. See Section 7.7 for the product
% string lists, one each for L2P, L3, L4, and GMPE
% products. See Section 7.7. \\ \hline

% <Additional Segregator> &
% Optional text to distinguish
% between files with the same
% <Product String>. Dashes are
% not allowed within this element. &
% This text is used since the other filename
% components are sometimes insufficient to
% uniquely identify a file. For example, in L2P or
% L3U (un-collated) products this is often the
% original file name or processing algorithm. Note,
% underscores should be used, not dashes. For L4
% files, this element should begin with the
% appropriate regional code as defined in Section 7.8. This component is optional but must be used
% in those cases were non-unique filenames would otherwise result. \\ \hline

% <GDS Version> & nn.n & Version number of the GDS used to process the file. For example, GDS 2.0 = “02.0”. \\ \hline
% <File Version> & xx.x & Version number for the file, for example, “01.0”. \\ \hline

% <File Type> & netCDF data file suffix (nc) or ISO metadata file suffix (xml) & 
% Indicates this is a netCDF file containing data or its corresponding ISO-19115 metadata record in XML.\\ \hline

% \end{longtable}

% \subsubsection{L2\_GHRSST Filename Example}
% 20070503132300-NAVO-L2P\_GHRSST-SSTblend-AVHRR17\_L-SST\_s0123\_e0135-v02.0-fv01.0.nc \par
% The above file contains GHRSST L2P blended SST data for 03 May 2007, from AVHRR LAC data
% collected from the NOAA-17 platform. The granule begins at 13:23:00 hours. It is version 1.0 of the
% file and was produced by the NAVO RDAC in accordance with the GDS 2.0. The <Additional
% Segregator> text is “SST\_s0123\_e0135”. \par

% \subsubsection{L3\_GHRSST Filename Example}
% 20070503110153-REMSS-L3C\_GHRSST-SSTsubskin-TMI-tmi\_20070503rt-v02.0-fv01.0.nc \par
% The above file was produced by the REMSS RDAC and contains collated L3 sub-skin SST data from
% the TMI instrument for 03 May 2007. The collated file has a centre time of at 11:01:53 hours. It is
% version 1.0 of the file and was produced according to GDS 2.0 specifications. Its <Additional
% Segregator> text is “tmi\_20070503rt”. \par

% \subsubsection{L4\_GHRSST Filename Example}
% 20070503120000-UKMO-L4\_GHRSST-SSTfnd-OSTIA-GLOB-v02.0-fv01.0.nc \par
% The above file contains L4 foundation SST data produced at the UKMO RDAC using the OSTIA
% system. It is global coverage, contains data for 03 May 2007, was produced to GDS 2.0 specifications
% and is version 1.0 of the file. The nominal time of the OSTIA analysis is 12:00:00 hours. \par


% \subsection{<Indicative Date>}
% The identifying date for this data set, using the format YYYYMMDD, where YYYY is the four-digit year,
% MM is the two-digit month from 01 to 12, and DD is the two-digit day of month from 01 to 31. The date
% used should best represent the observation date for the dataset. \par

% \subsection{<Indicative Time>}
% The identifying time for this data set in UTC, using the format HHMMSS, where HH is the two-digit
% hour from 00 to 23, MM is the two-digit minute from 00 to 59, and SS is the two-digit second from 00 to
% 59. The time used is dependent on the <Processing Level> of the data set: \par \vspace{0.5in}

% L2P: start time of granule \\
% L3U: start time of granule \\ 
% L3C and L3S: centre time of the collation window \\ 
% L4 and GMPE: nominal time of analysis \par \vspace{0.1in}
% All times should be given in UTC and should be chosen to best represent the observation time for this
% dataset. Note: RDACs should ensure the applications they use to determine UTC proprerly account
% for leap seconds.

% \subsection{<RDAC>}
% Codes used for GHRSST Regional Data Assembly Centres (RDACs) are provided in the table below.
% New codes are assigned by the GHRSST Data And Systems Technical Advisory Group (DAS-TAG)
% and entered into the table upon agreement by the GDAC, LTSRF, and relevant RDACs. \par


% % Table 7-2: Regional Data Assembly Centre (RDAC) code table.
% \begin{table}[h]
% \centering
% \caption{Regional Data Assembly Centre (RDAC) code table}
% \label{tab:RDAC code table}
% \begin{tabular}{|p{0.2\textwidth}|p{0.8\textwidth}|}
% \hline
% \rowcolor{lightgray}
% RDAC Code & GHRSST RDAC Name \\ \hline
% ABOM & Australian Bureau of Meteorology \\ \hline
% CMC & Canadian Meteorological Centre \\ \hline
% DMI & Danish Meteorological Institute \\ \hline
% EUR & European RDAC \\ \hline
% GOS & Gruppo di Oceanografia da Satellite \\ \hline
% JPL & JPL Physical Oceanography Distributed Active Archive Center \\ \hline
% JPL\_OUROCEAN & JPL OurOcean Project \\ \hline
% METNO & Norwegian Meteorological Institute \\ \hline
% MYO & MyOcean \\ \hline
% NAVO & Naval Oceanographic Office \\ \hline
% NCDC & NOAA National Climatic Data Center \\ \hline
% NEODAAS & NERC Observation Data Acquisition and Analysis Service \\ \hline
% NOC & National Oceanography Centre, Southampton \\ \hline
% NODC & NOAA National Oceanographic Data Center \\ \hline
% OSDPD & NOAA Office of Satellite Data Processing and Distribution \\ \hline
% OSISAF & EUMETSAT Ocean and Sea Ice Satellite Applications Facility \\ \hline
% REMSS & Remote Sensing Systems, CA, USA \\ \hline
% RSMAS & University of Miami, RSMAS \\ \hline
% UKMO & UK Meteorological Office \\ \hline
% UPA & United Kingdom Multi-Mission Processing and Archiving Facility \\ \hline
% ESACCI & ESA SST Climate Change Initiative \\ \hline
% JAXA & Japan Aerospace Exploration Agency \\ \hline
% New codes & Please contact the GHRSST international Project Office if you require new
% codes to be included in future revisions of the GDS. \\ \hline

% \end{tabular}
% \end{table}


% \subsection{<Processing Level>}
% Satellite data processing level definitions can lead to ambiguous situations, especially regarding the
% distinction between L3 and L4 products. GHRSST identified the use of analysis procedures to fill gaps
% where no observations exist to resolve this ambiguity. Within GHRSST filenames, the <Processing
% Level> codes are shown below in Table 7-3. GHRSST currently establishes standards for L2P, L3U,
% L3C, L3S, and L4 (GHRSST Multi-Product Ensembles known as GMPE are a special kind of L4
% product for which GHRSST also provides standards). \par

% % Table 7-3:  GHRSST Processing Level Conventions and Codes
% \begin{table}[h]
% %\centering
% \caption{GHRSST Processing Level Conventions and Codes}
% \label{tab:GHRSST Processing Level Conventions and Codes}
% \begin{tabular}{|p{0.1\textwidth}|p{0.15\textwidth}|p{0.75\textwidth}|}
% \hline
% \rowcolor{lightgray}
% \textbf{Level} & \textbf{<Processing Level> Code} & \textbf{Description} \\ \hline

% Level 0 & L0 &
% Unprocessed instrument and payload data at full resolution. GHRSST
% does not make recommendations regarding formats or content for data
% at this processing level. \\ \hline

% Level 1A & L1A & 
% Reconstructed unprocessed instrument data at full resolution, time
% referenced, and annotated with ancillary information, including
% radiometric and geometric calibration coefficients and geo-referencing
% parameters, computed and appended, but not applied, to L0 data.
% GHRSST does not make recommendations regarding formats or
% content for data at this processing level. \\ \hline

% Level 1B & L1B & 
% Level 1A data that have been processed to sensor units. GHRSST
% does not currently make recommendations regarding formats or content
% for L1B data. \\ \hline

% Level 2 & Preprocessed L2P &
% Geophysical variables derived from Level 1 source data at the same
% resolution and location as the Level 1 data, typically in a satellite
% projection with geographic information. These data form the
% fundamental basis for higher-level GHRSST products and require
% ancillary data and uncertainty estimates. \\ \hline

% Level 3 & L3U L3C L3S & Level 2 variables mapped on a defined grid with reduced requirements for ancillary data. 
% Uncertainty estimates are still mandatory. Three
% types of L3 products are defined:
% \begin{itemize}
% \item{Un-collated (L3U): L2 data granules remapped to a space grid without combining any observations from overlapping orbits}
% \item {Collated (L3C): observations combined from a single instrument into a space-time grid}
% \item {Super-collated (L3S): observations combined from multiple instruments into a space-time grid.}
% \end{itemize}
% Note that L3 GHRSST products do not use analysis or interpolation
% procedures to fill gaps where no observations are available. \\ \hline

% Level 4 & L4 &
% Data sets created from the analysis of lower level data that result in
% gridded, gap-free products. SST data generated from multiple sources
% of satellite data using optimal interpolation are an example of L4
% GHRSST products. GMPE products are a type of L4 dataset. \\  \hline

% Note that within GHRSST, all L2P files require a full set of extensive ancillary data such as wind
% speeds and times of observation that are provided as \'dynamic flags\' that users can manipulate to
% filter data according to their own quality criteria. L2P files form the basis of higher-level products and
% are often the best level products for data assimilation. The requirement for dynamic flags is particularly
% important in this context. Higher-level L3 products are often intended for general use or created for
% input to Level 4 analysis systems so the requirement for extensive ancillary data is reduced. Since
% some GHRSST RDACs only process data natively on grids (especially in the case of geostationary
% platform observations), the GDS 2.0 L3 specification is flexible enough to allow for the creation of L3
% files which meet all the content requirements of a L2P file. In all L2P and L3 cases, bias and standard
% deviation uncertainty estimates are mandatory. \par

% The distinction between L3 GHRSST and L4 GHRSST data is made primarily on whether or not any
% gap-filling techniques are employed, not on whether data from multiple instruments is used in the L3
% product. If no gap filling procedure (such as optimal interpolation) is used, then the product remains a
% L3 GHRSST product. GHRSST defines three kinds of L3 files: un-collated (L3U), collated (L3C), and
% super-collated (L3S). If gap filling is used to fill all observations gaps, then the resulting gap-free data
% are considered L4 GHRSST data products. \par
% \end{tabular}
% \end{table}

%%% Arret ici !!!!!!

% \subsection{<SST Type>}
% In conjunction with the NetCDF Climate and Forecast (CF) community [AD-9] the GHRSST Science
% Team agreed on the CF standard names for “SST” shown in the following figure and described in
% more detail below. The names were first included in CF-1.3, and the current version (CF-1.4) of the
% standard name table that can be found in [AD-8]. In addition, the GHRSST Science Team agreed to
% use the CF Naming Convention [AD-3] for variable names that do not already exist as part of the CF
% Convention. CF definitions are used in the GDS and across GHRSST and are shown schematically in
% Figure 7-1. The different kinds of SST are detailed later in this section and the relevant <SST Type>
% codes to be used in the filenames are provided. \par \vspace{1in}

% % Figure 7-1. Overview of SST measurement types used within GHRSST
% \begin{figure}[h]
%     \caption{Overview of SST measurment types used within GHRSST}
%     \label{fig:Overview of SST measurment types used within GHRSST}
% \end{figure}
% \textbf{Figure 7-1. Overview of SST measurement types used within GHRSST.} \par
% CF Definition: \emph{sea\_surface\_temperature is usually abbreviated as "SST". It is the temperature of sea
% water near the surface (including the part under sea-ice, if any), and not the interface temperature,
% whose standard name is surface\_temperature. For the temperature of sea water at a particular depth
% or layer, a data variable of sea\_water\_temperature with a vertical coordinate axis should be used.} \par
% Additional Details: The sea surface skin temperature (SSTskin) as defined above represents the
% actual temperature of the water across a very small depth of approximately 20 micrometers. This
% definition is chosen for consistency with the majority of infrared satellite and ship mounted radiometer
% measurements. \par

% \textbf{Sea\_surface\_subskin\_temperature (GHRSST <SST Type>: SSTsubskin):}\par
% CF Definition: \emph{The surface called "surface" means the lower boundary of the atmosphere. The sea
% surface subskin temperature is the temperature at the base of the conductive laminar sub-layer of the
% ocean surface, that is, at a depth of approximately 1 - 1.5 millimetres below the air-sea interface. For
% practical purposes, this quantity can be well approximated to the measurement of surface temperature
% by a microwave radiometer operating in the 6 - 11 gigahertz frequency range, but the relationship is
% neither direct nor invariant to changing physical conditions or to the specific geometry of the
% microwave measurements. Measurements of this quantity are subject to a large potential diurnal cycle
% due to thermal stratification of the upper ocean layer in low wind speed high solar irradiance
% conditions.}\par
% Additional Details: The sea surface subskin temperature (SSTsubskin) represents the temperature at
% the base of the thermal skin layer. The difference between SSTint and SSTsubskin is related to the
% net flux of heat through the thermal skin layer. SSTsubskin is the temperature of a layer approximately
% 1 mm thick at the ocean surface.\par

% \textbf{Sea\_water\_temperature (GHRSST <SST Type>: SSTdepth or SSTz):}\par
% CF Definition: \emph{The general term, "bulk" sea surface temperature, has the standard name
% sea\_surface\_temperature with no associated vertical coordinate axis. The temperature of sea water at
% a particular depth (other than the foundation level) should be reported using the standard name
% sea\_water\_temperature and, wherever possible, supplying a vertical coordinate axis or scalar
% coordinate variable.}\par
% Additional Details: Sea water temperature (SSTdepth or SSTz, for example SST1.5m) is the terminology
% adopted by GHRSST to represent in situ measurements near the surface of the ocean that have
% traditionally been reported simply as SST or "bulk" SST. For example SST6m would refer to an SST
% measurement made at a depth of 6 m. Without a clear statement of the precise depth at which the
% SST measurement was made, and the circumstances surrounding the measurement, such a sample
% lacks the information needed for comparison with, or validation of satellite-derived estimates of SST
% using other data sources. The terminology has been introduced to encourage the reporting of depth
% (z) along with the temperature.\par
% All measurements of water temperature beneath the SSTsubskin are obtained from a wide variety of
% sensors such as drifting buoys having single temperature sensors attached to their hull, moored buoys
% that sometimes include deep thermistor chains at depths ranging from a few meters to a few thousand
% meters, thermosalinograph (TSG) systems aboard ships recording at a fixed depth while the vessel is 
% underway, Conductivity Temperature and Depth (CTD) systems providing detailed vertical profiles of
% the thermohaline structure used during hydrographic surveys and to considerable depths of several
% thousand meters, and various expendable bathythermograph systems (XBT). In all cases, these
% temperature observations are distinct from those obtained using remote sensing techniques and
% measurements at a given depth should be referred to as sea\_water\_temperature qualified by a depth
% in meters rather than sea surface temperatures. The situation is complicated further when one
% considers ocean model outputs for which the SST may be the mean SST over a layer of the ocean
% several tens of meters thick.\par

% \textbf{Sea\_surface\_foundation\_temperature (GHRSST <SST Type>: SSTfnd):}\par
% CF Definition: \emph{The surface called "surface" means the lower boundary of the atmosphere. The sea
% surface foundation temperature is the water temperature that is not influenced by a thermally stratified
% layer of diurnal temperature variability (either by daytime warming or nocturnal cooling). The
% foundation temperature is named to indicate that it is the temperature from which the growth of the
% diurnal thermocline develops each day, noting that on some occasions with a deep mixed layer there
% is no clear foundation temperature in the surface layer. In general, sea surface foundation temperature
% will be similar to a night-time minimum or pre-dawn value at depths of between approximately 1 and 5
% meters. In the absence of any diurnal signal, the foundation temperature is considered equivalent to
% the quantity with standard name sea\_surface\_subskin\_temperature. The sea surface foundation
% temperature defines a level in the upper water column that varies in depth, space, and time depending
% on the local balance between thermal stratification and turbulent energy and is expected to change
% slowly over the course of a day. If possible, a data variable with the standard name
% sea\_surface\_foundation\_temperature should be used with a scalar vertical coordinate variable to
% specify the depth of the foundation level. Sea surface foundation temperature is measured at the base
% of the diurnal thermocline or as close to the water surface as possible in the absence of thermal
% stratification. Only in situ contact thermometry is able to measure the sea surface foundation
% temperature. Analysis procedures must be used to estimate sea surface foundation temperature value
% from radiometric satellite measurements of the quantities with standard names
% sea\_surface\_skin\_temperature and sea\_surface\_subskin\_temperature. Sea surface foundation
% temperature provides a connection with the historical concept of a "bulk" sea surface temperature
% considered representative of the oceanic mixed layer temperature that is typically represented by any
% sea temperature measurement within the upper ocean over a depth range of 1 to approximately 20
% meters. The general term, "bulk" sea surface temperature, has the standard name
% sea\_surface\_temperature with no associated vertical coordinate axis. Sea surface foundation
% temperature provides a more precise, well-defined quantity than "bulk" sea surface temperature and,
% consequently, is more representative of the mixed layer temperature. The temperature of sea water at
% a particular depth (other than the foundation level) should be reported using the standard name
% sea\_water\_temperature and, wherever possible, supplying a vertical coordinate axis or scalar
% coordinate variable.}\par
% Additional Details: Through the definition of the CF standard names, GHRSST is attempting to
% discourage the use of the term “bulk SST”, replacing it instead with sea\_water\_temperature
% (SSTdepth) and a depth coordinate, or sea\_surface\_foundation\_temperature (SSTfnd) and a depth
% coordinate if possible, if the observation comes from the base of the diurnal thermocline.\par

% \textbf{Blended SST (GHRSST <SST Type>: SSTblend):}\par
% In addition to the CF standard names defined above, GHRSST also uses the term “Blended SST” for
% ambiguous cases when the depth or type of SST is not well known. This ambiguity in depth may arise
% in some L4 analysis products that merge multiple types of SST from satellite and in situ observations.
% Note, however, that many L4 analysis systems do attempt to specifically create a sea surface
% foundation temperature, SSTfnd.\par
% The SST codes and CF standard names defined above and used within GHRSST are summarized
% along with their key characteristics in Table 7-4.\par

% %Table 7-4. GHRSST <SST Type> code and summary table
% \begin{table}[h]
%     \caption{ECCO <SST Type> code and summary table}
%     \label{tab:ECCO <SST Type> code and summary table}
%     \begin{tabular}{|p{0.1\textwidth}|p{0.65\textwidth}|p{0.1\textwidth}|p{0.15\textwidth}|}
%     \rowcolor{lightgray}
%     \textbf{ECCO <SST Type>} & \textbf{CF Standard Name} & \textbf{Approximate Depth} & \textbf{Typically Observed
%     by...} \\ \hline
%     SSTint & \texttt{sea\_surface\_temperature} &  0 meters & Not presently measureable \\ \hline
%     SSTskin & \texttt{sea\_surface\_skin\_temperature} & 10 - 20 micrometers & Infrared radiometers
%     operating in a range
%     of wavelengths form
%     3.7 to 12
%     micrometers \\ \hline

%     SSTsubskin & \texttt{sea\_surface\_subskin\_temperature} & 1 - 1.5 millimeters & Microwave radiometers operating in a range
%     of frequencies from
%     6-11 gigahertz \\ \hline

%     SSTdepth & \texttt{sea\_water\_temperature} & Specified by
%     vertical
%     coordinate
%     (e.g., SST\_{5m}) & In situ observing systems \\ \hline

%     SSTfnd & \texttt{sea\_surface\_foundation\_temperature} & 1-5 meters pre-dawn & 
%     In situ observing systems \\ \hline

%     SSTblend & None &  Unknown & Blend of satellite and in situ observations \\ \hline
    
%     \end{tabular}
% \end{table}


% \subsection{<Product String>}
% The current set of GHRSST product strings is listed in tables below, in one table each for L2P (Table
% 7-5), L3 (Table 7-6), L4 (Table 7-7) and GMPE (Table 7-8) products. Included in the L2P table are
% also codes for satellite platforms and sensors. New strings are entered into the tables upon
% registration by the DAS-TAG and agreement by the GDAC, LTSRF, and relevant RDACs. These
% product strings are used within the GHRSST filename convention and within the GHRSST unique data
% set codes described in Section 7.9. The satellite platform and satellite sensor entries are also used in
% the netCDF global attributes, \texttt{platform} and \texttt{sensor}, for all GHRSST product files. See Section 8.2
% for more information on the required \texttt{global attributes}. \par

