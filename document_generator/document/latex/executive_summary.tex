\pagebreak\section{Executive Summary}
\par \vspace{0.5cm}

The ECCO Version 4 Release 4 (V4r4) products are comprehensive global ocean and sea-ice state estimates spanning from 1992 to 2018. These datasets are dynamically and kinematically consistent reconstructions of the three-dimensional, time-evolving ocean, sea-ice, and surface atmospheric states. They include a wide range of variables such as temperature, salinity, velocity, sea level anomalies, and fluxes (e.g., heat, freshwater, and salt). The data are available at daily, monthly, and instantaneous intervals on both the high-resolution LLC90 grid and a 0.5-degree interpolated grid. The datasets adhere to modern metadata standards and are formatted in netCDF-4 (and may be available in other data format such as \hyperlink{https://en.wikipedia.org/wiki/Zarr_(data_format)}{Zarr data format}) for accessibility via NASA's Earthdata Cloud infrastructure ( \hyperlink{https://podaac.jpl.nasa.gov/}{PO.DAAC-NASA}).

The key features of ECCO V4r4 include its ability to assimilate diverse observational datasets from satellites and in situ measurements using the MIT general circulation model (MITgcm). This allows for accurate representation of global ocean dynamics and climate processes. The release also introduces advanced cloud-native services for efficient data access and processing. These products are essential for studying ocean circulation, climate variability, sea-level changes, and freshwater fluxes on a global scale.

This user guide document first provides the scope and the key overall content of the present document, then an overview of ECCO data products (Level 4) followed by detailed technical specifications of dataset filename, file structure and supporting configuration. Finaly, a full description of grid geometry, the coordinates and the data variables for both native lat-lon-cap 90 (llc90), latlon 0.5-degree and 1D datasets are provided with example of each of them.



% A new generation of integrated Sea Surface Temperature (SST) data products are being provided by the Group for High Resolution Sea Surface Temperature (GHRSST). L2 products are provided by a variety of data providers in a common format. L3 and L4 products combine, in near-real time, various SST data products from several different satellite sensors and in situ observations and maintain fine spatial and temporal resolution needed by SST inputs to a variety of ocean and atmosphere applications in the operational and scientific communities. Other GHRSST products provide diagnostic data sets and global multi-product ensemble analysis products. Retrospective reanalysis products are provided in a non real time critical offline manner. All GHRSST products have a standard format, include uncertainty estimates for each measurement, and are served to the international user community free of charge through a variety of data transport mechanisms and access points that are collectively referred to as the GHRSST Regional/Global Task Sharing (R/GTS) framework. 
% \par \vspace{0.5cm}
% \noindent The GHRSST Data Specification (GDS) Version 2.0 is a technical specification of GHRSST products and services. It consists of a technical specification document (this volume) and a separate Interface Control Document (ICD). The GDS technical documents are supported by a User Manual and a complete description of the GHRSST ISO-19115-2 metadata model. GDS-2.0 represents a consensus opinion of the GHRSST international community on how to optimally combine satellite and in situ SST data streams within the R/GTS. The GDS also provides guidance on how data providers might implement SST processing chains that contribute to the R/GTS.
% \par \vspace{0.5cm}
% \noindent This document first provides an overview of GHRSST followed by detailed technical specifications of the adopted file naming specification and supporting definitions and conventions used throughout GHRSST and the technical specifications for all GHRSST Level 2P, Level 3, Level 4, and GHRSST Multi-Product Ensemble data products. In addition, the GDS 2.0 Technical Specification provides controlled code tables and best practices for identifying sources of SST and ancillary data that are used within GHRSST data files.
% \par \vspace{0.5cm}
% \noindent The GDS document has been developed for data providers who wish to produce any level of GHRSST data product and for all users wishing to fully understand GHRSST product conventions, GHRSST data file contents, GHRSST and Climate Forecast definitions for SST, and other useful information. For a complete discussion and access to data products and services see https://www.ghrsst.org, which is a central portal for all GHRSST activities.
