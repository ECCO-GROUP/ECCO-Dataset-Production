\pagebreak
\section{GDS 2.0 Data Product File Structure}
\subsection{Overview of the GDS 2.0 netCDF File Format}

GDS 2.0 data files preferentially use the \textbf{netCDF-4 Classic} format. However, as netCDF-4 is a
relatively new format and includes a significant number of new features that may not be well supported
by existing user applications and tools, the GHRSST Science Team agreed to support both netCDF-3
and netCDF-4 format data files during a transition period. At the 11th GHRSST Science Team
meeting, Lima Peru, 21-25th June 2010 it was agreed that the transition period would end in 2013 at
which point (subject to positive developments in the user community using netCDF-4) the use of
netCDF-3 format data products will cease within the GHRSST R/GTS framework. \textbf{NetCDF-3 data
products shall be delivered to the GDAC with an accompanying MMR file records as described
in Section 13}. While netCDF-3 can store the metadata, it is computationally expensive to extract it
from externally-compressed netCDF-3 files. A major advantage to the use of NetCDF-4 format
products from the producer's perspective is that no additional metadata records are required when
using this format since the GDAC and LTSRF can easily extract it from the files without having to
decompress the entire file. \par \vspace{0.1in}


These GDS 2.0 formatted data sets must comply with the Climate and Forecast (CF) Conventions,
v1.4 [AD-3] or later because these conventions provide a practical standard for storing oceanographic
data in a robust, easily-preserved for the long-term, and interoperable manner. The CF-compliant
netCDF data format is flexible, self-describing, and has been adopted as a de facto standard for many
operational and scientific oceanography systems. Both netCDF and CF are actively maintained
including significant discussions and inputs from the oceanographic community (see \url{http://cfpcmdi.llnl.gov/discussion/index_html}). 
The CF convention generalizes and extends the Cooperative
Ocean/Atmosphere Research Data Service (COARDS, [AD-4]) Convention but relaxes the COARDS
constraints on dimension order and specifies methods for reducing the size of datasets. The purpose
of the CF Conventions is to require conforming datasets to contain sufficient metadata so that they are
self-describing, in the sense that each variable in the file has an associated description of what it
represents, physical units if appropriate, and that each value can be located in space (relative to earthbased coordinates) and time. In addition to the CF Conventions, GDS 2.0 formatted files follow some
of the recommendations of the Unidata Attribute Convention for Dataset Discovery (ACDD, [AD-7]). \par \vspace{0.1in}

In the context of netCDF, a variable refers to data stored in the file as a vector or as a
multidimensional array. Each variable in a GHRSST netCDF file consists of a 2-dimensional [i x j], 3-
dimensional [i x j x k], or 4-dimensional [i x j x k x l] array of data. The dimensions of each variable
must be explicitly declared in the dimension section. \par \vspace{0.1in}

Within the netCDF file, global attributes are used to hold information that applies to the whole file, such
as the data set title. Each individual variable must also have its own attributes, referred to as variable
attributes. These variable attributes define, for example, an offset, scale factor, units, a descriptive
version of the variable name, and a fill value, which is used to indicate array elements that do not
contain valid data. Where applicable, SI units should be used and described by a character string,
which is compatible with the Unidata UDUNITS-2 package [AD-5]. \par \vspace{0.1in}

All GHRSST GDS 2.0 files conform to this structure and share a common set of netCDF global
attributes. These global attributes include those required by the CF Convention plus additional ones
required by the GDS 2.0. The required set of global attributes is described in Section 8.2 and entities
within the GHRSST R/GTS framework are free to add their own, as long as they do not contradict the
GDS 2.0 and CF requirements. \par \vspace{0.1in}

Following the CF convention, each variable also has a set of variable attributes. The required variable
attributes are described in Section 8.3. In a few cases, some of these variable attributes may not be
relevant for certain variables or additional variable attributes may be required. In those cases, the
variable descriptions in each of the L2P, L3, L4, and GMPE product specifications (Sections 9, 10, 11,
and 12) will identify the differences and specify requirements for each product. As with the global
attributes, entities within the GHRSST R/GTS framework are free to add their own variable attributes,
as long as they do not contradict the GDS 2.0 and CF requirements. \par \vspace{0.1in}

While the exact volumes can vary, an average L2P file will use about 33 bytes per pixel, an L3 file 28
bytes per pixel, and an L4 file about 8 bytes per pixel. The data type encodings for each variable are
fixed except for the experimental fields, which are flexible and can chosen by the producing RDAC. \par \vspace{0.1in}

\subsection{GDS 2.0 netCDF Global Attributes}
Table 8-1 below summarizes the global attributes that are mandatory for every GDS 2.0 netCDF data
file. More details on the CF-mandated attributes (as indicated in the Source column) are available at:
\url{http://cf-pcmdi.llnl.gov/documents/cf-conventions/1.4/cf-conventions.html#attribute-appendix} and
information on the ACDD recommendations is available at
\url{http://www.unidata.ucar.edu/software/netcdf-java/formats/DataDiscoveryAttConvention.html}.
\par \vspace{0.1in}

% Table 8-1 Mandatory global attributes for GDS 2.0 netCDF data files
% \begin{longtable}{|p{0.3\textwidth}|p{0.1\textwidth}|p{0.5\textwidth}|p{0.1\textwidth}|}
% \caption{Mandatory global attributes for GDS 2.0 netCDF data files} \label{tab:global-attributes} \\
% \hline \endhead
% \hline \endfoot
% \rowcolor{lightgray} \textbf{Global Attribute Name} & \textbf{Source} & \textbf{Description} & \textbf{Type} \\ \hline
%     % Table 8-1 Mandatory global attributes for GDS 2.0 netCDF data files
\begin{longtable}{|p{0.276\textwidth}|p{0.092\textwidth}|p{0.46\textwidth}|p{0.092\textwidth}|}
\caption{Mandatory global attributes for GDS 2.0 netCDF data files}
\label{tab:global-attributes} \\ 
\hline \endhead
\hline \endfoot
\rowcolor{lightgray} \textbf{Global Attribute Name} & \textbf{Type} & \textbf{Description} & \textbf{Source} \\ \hline
\rowcolor{LightCyan} acknowledgement & string & A place to acknowledge various types of support for the project that produced this data. & ACDD \\ \hline

\rowcolor{LightCyan} cdm\_data\_type & string & The data type, as derived from Unidata's Common Data Model Scientific Data types and understood by THREDDS. (This is a THREDDS "dataType", and is different from the CF NetCDF attribute 'featureType', which indicates a Discrete Sampling Geometry file in CF.) & ACDD \\ \hline

\rowcolor{LightCyan} comment & string & Miscellaneous information about the data, not captured elsewhere. This attribute is defined in the CF Conventions. & CF, ACDD \\ \hline

\rowcolor{LightCyan} conventions & string & A text string identifying the netCDF conventions followed (e.g., CF-1.4, ACDD 1-3). &  \\ \hline

\rowcolor{LightCyan} creator\_email & string & The email address of the person (or other creator type specified by the creator\_type attribute) principally responsible for creating this data. & ACDD \\ \hline

\rowcolor{LightCyan} creator\_name & string & The name of the person (or other creator type specified by the creator\_type attribute) principally responsible for creating this data. & ACDD \\ \hline

\rowcolor{LightCyan} creator\_url & string & The URL of the of the person (or other creator type specified by the creator\_type attribute) principally responsible for creating this data. & ACDD \\ \hline

\rowcolor{LightCyan} date\_created & string & The date on which this version of the data was created. & ACDD \\ \hline

\rowcolor{LightCyan} easternmost\_longitude & float & Decimal degrees east, range -180 to +180. This is equivalent to ACDD geospatial\_lon\_max. & podaac \\ \hline

\rowcolor{LightCyan} geospatial\_lat\_resolution & float & Latitude Resolution in units matching geospatial\_lat\_units. & ACDD \\ \hline

\rowcolor{LightCyan} geospatial\_lat\_units & string & Units of the latitudinal resolution. Typically "degrees\_north" & ACDD \\ \hline

\rowcolor{LightCyan} geospatial\_lon\_resolution & float & Longitude Resolution in units matching geospatial\_lon\_resolution & ACDD \\ \hline

\rowcolor{LightCyan} geospatial\_lon\_units & string & Units of the longitudinal resolution. Typically "degrees\_east" & ACDD \\ \hline

\rowcolor{LightCyan} history & string & The name of the institution principally responsible for originating this data. This attribute is recommended by the CF convention. & CF, ACDD \\ \hline

\rowcolor{LightCyan} id & string & An identifier for the data set, provided by and unique within its naming authority. The combination of the "naming authority" and the "id" should be globally unique, but the id can be globally unique by itself also. IDs can be URLs, URNs, DOIs, meaningful text strings, a local key, or any other unique string of characters. The id should not include white space characters. & ACDD \\ \hline

\rowcolor{LightCyan} institutions & string & The name of the institution principally responsible for originating this data. This attribute is recommended by the CF convention. & CF, ACDD \\ \hline

\rowcolor{LightCyan} keywords & string & GCMD Science Keyword(s) & ACDD \\ \hline

\rowcolor{LightCyan} keywords\_vocabulary & string & The unique name or identifier of the vocabulary from which keywords are taken. e.g., the NASA Global Change Master Directory (GCMD) Science Keywords. & ACDD \\ \hline

\rowcolor{LightCyan} license & string & Provide the URL to a standard or specific license, enter "Freely Distributed" or "None", or describe any restrictions to data access and distribution in free text. & ACDD \\ \hline

\rowcolor{LightCyan} Metadata\_Conventions & string & A comma-separated list of the conventions that are followed by the dataset.  & ACDD \\ \hline

\rowcolor{LightCyan} metadata\_link & string & Link to collection metadata record at archive & ACDD \\ \hline

\rowcolor{LightCyan} naming\_authority & string & The organization that provides the initial id (see above) for the dataset. The naming authority should be uniquely specified by this attribute via reverse-DNS naming convention. & ACDD \\ \hline

\rowcolor{LightCyan} netcdf\_version\_id  & string & Version of netCDF libraries used to create this file. For example, "4.1.1" & GDS \\ \hline

\rowcolor{LightCyan} northernmost\_latitude & float & Decimal degrees north, range -90 to +90. This is equivalent to ACDD geospatial\_lat\_max. & GDS \\ \hline

\rowcolor{LightCyan} processing\_level & string & A textual description of the processing (or quality control) level of the data. & ACDD \& GDS \\ \hline

\rowcolor{LightCyan} product\_version & string & The product version of this data file & GDS \\ \hline

\rowcolor{LightCyan} project & string & The name of the project(s) principally responsible for originating this data. & ACDD \\ \hline

\rowcolor{LightCyan} publisher\_email & string & The email address of the person (or other entity specified by the publisher\_type attribute) responsible for publishing the data file or product to users, with its current metadata and format. & ACDD \\ \hline

\rowcolor{LightCyan} publisher\_name & string & The name of the person (or other entity specified by the publisher\_type attribute) responsible for publishing the data file or product to users, with its current metadata and format. & ACDD \\ \hline

\rowcolor{LightCyan} publisher\_url & string & The URL of the person (or other entity specified by the publisher\_type attribute) responsible for publishing the data file or product to users, with its current metadata and format. & ACDD \\ \hline

\rowcolor{LightCyan} references & string & Published or web-based references that describe the data or methods used to produce it. Recommend URIs (such as a URL or DOI) for papers or other references. This attribute is defined in the CF conventions. & ACDD \\ \hline

\rowcolor{LightCyan} source & string & Method of production of the original data. & CF \\ \hline

\rowcolor{LightCyan} sourthernmost\_latitude & float & Decimal degrees north, range -90 to +90. This is equivalent to ACDD geospatial\_lat\_min. & GDS \\ \hline

\rowcolor{LightCyan} spatial\_resolution & string & A string describing the approximate resolution of the product. & GDS \\ \hline

\rowcolor{LightCyan} standard\_name\_vocabulary & string & The name and version of the controlled vocabulary from which variable standard names are taken. & ACDD \\ \hline

\rowcolor{LightCyan} start\_time & string & Representative date and time of the end of the granule in the ISO 8601 compliant format of "yyyymmddThhmmssZ". & GDS \\ \hline

\rowcolor{LightCyan} stop\_time & string & Representative date and time of the end of the granule in the ISO 8601 compliant format of "yyyymmddThhmmssZ". & GDS \\ \hline

\rowcolor{LightCyan} summary & string & A paragraph describing the dataset, analogous to an abstract for a paper. & ACDD \\ \hline

\rowcolor{LightCyan} time\_coverage\_end & string & Identical to stop\_time. Included for increased ACDD compliance. & ACDD \\ \hline

\rowcolor{LightCyan} time\_coverage\_start & string & Identical to start\_time. Included for increased ACDD compliance. & ACDD \\ \hline

\rowcolor{LightCyan} title & string & A short phrase or sentence describing the dataset. In many discovery systems, the title will be displayed in the results list from a search, and therefore should be human readable and reasonable to display in a list of such names. This attribute is recommended by the NetCDF Users Guide (NUG) and the CF conventions. & CF, ACDD \\ \hline

\rowcolor{LightCyan} uuid & string & A Universally Unique Identifier (UUID). Numerous, simple tools can be used to create a UUID, which is inserted as the value of this attribute. See http://en.wikipedia.org/wiki/Universally\_Unique\_Identifier for more information and tools. & GDS \\ \hline

\rowcolor{LightCyan} westernmost\_longitude & float & Decimal degrees east, range -180 to +180. This is equivalent to ACDD geospatial\_lon\_min. & GDS \\ \hline

\end{longtable} % inserting contents from a python generated tex file
% \end{longtable}
% Table 8-1 Mandatory global attributes for GDS 2.0 netCDF data files
\begin{longtable}{|p{0.276\textwidth}|p{0.092\textwidth}|p{0.46\textwidth}|p{0.092\textwidth}|}
\caption{Mandatory global attributes for GDS 2.0 netCDF data files}
\label{tab:global-attributes} \\ 
\hline \endhead
\hline \endfoot
\rowcolor{lightgray} \textbf{Global Attribute Name} & \textbf{Type} & \textbf{Description} & \textbf{Source} \\ \hline
\rowcolor{LightCyan} acknowledgement & string & A place to acknowledge various types of support for the project that produced this data. & ACDD \\ \hline

\rowcolor{LightCyan} cdm\_data\_type & string & The data type, as derived from Unidata's Common Data Model Scientific Data types and understood by THREDDS. (This is a THREDDS "dataType", and is different from the CF NetCDF attribute 'featureType', which indicates a Discrete Sampling Geometry file in CF.) & ACDD \\ \hline

\rowcolor{LightCyan} comment & string & Miscellaneous information about the data, not captured elsewhere. This attribute is defined in the CF Conventions. & CF, ACDD \\ \hline

\rowcolor{LightCyan} conventions & string & A text string identifying the netCDF conventions followed (e.g., CF-1.4, ACDD 1-3). &  \\ \hline

\rowcolor{LightCyan} creator\_email & string & The email address of the person (or other creator type specified by the creator\_type attribute) principally responsible for creating this data. & ACDD \\ \hline

\rowcolor{LightCyan} creator\_name & string & The name of the person (or other creator type specified by the creator\_type attribute) principally responsible for creating this data. & ACDD \\ \hline

\rowcolor{LightCyan} creator\_url & string & The URL of the of the person (or other creator type specified by the creator\_type attribute) principally responsible for creating this data. & ACDD \\ \hline

\rowcolor{LightCyan} date\_created & string & The date on which this version of the data was created. & ACDD \\ \hline

\rowcolor{LightCyan} easternmost\_longitude & float & Decimal degrees east, range -180 to +180. This is equivalent to ACDD geospatial\_lon\_max. & podaac \\ \hline

\rowcolor{LightCyan} geospatial\_lat\_resolution & float & Latitude Resolution in units matching geospatial\_lat\_units. & ACDD \\ \hline

\rowcolor{LightCyan} geospatial\_lat\_units & string & Units of the latitudinal resolution. Typically "degrees\_north" & ACDD \\ \hline

\rowcolor{LightCyan} geospatial\_lon\_resolution & float & Longitude Resolution in units matching geospatial\_lon\_resolution & ACDD \\ \hline

\rowcolor{LightCyan} geospatial\_lon\_units & string & Units of the longitudinal resolution. Typically "degrees\_east" & ACDD \\ \hline

\rowcolor{LightCyan} history & string & The name of the institution principally responsible for originating this data. This attribute is recommended by the CF convention. & CF, ACDD \\ \hline

\rowcolor{LightCyan} id & string & An identifier for the data set, provided by and unique within its naming authority. The combination of the "naming authority" and the "id" should be globally unique, but the id can be globally unique by itself also. IDs can be URLs, URNs, DOIs, meaningful text strings, a local key, or any other unique string of characters. The id should not include white space characters. & ACDD \\ \hline

\rowcolor{LightCyan} institutions & string & The name of the institution principally responsible for originating this data. This attribute is recommended by the CF convention. & CF, ACDD \\ \hline

\rowcolor{LightCyan} keywords & string & GCMD Science Keyword(s) & ACDD \\ \hline

\rowcolor{LightCyan} keywords\_vocabulary & string & The unique name or identifier of the vocabulary from which keywords are taken. e.g., the NASA Global Change Master Directory (GCMD) Science Keywords. & ACDD \\ \hline

\rowcolor{LightCyan} license & string & Provide the URL to a standard or specific license, enter "Freely Distributed" or "None", or describe any restrictions to data access and distribution in free text. & ACDD \\ \hline

\rowcolor{LightCyan} Metadata\_Conventions & string & A comma-separated list of the conventions that are followed by the dataset.  & ACDD \\ \hline

\rowcolor{LightCyan} metadata\_link & string & Link to collection metadata record at archive & ACDD \\ \hline

\rowcolor{LightCyan} naming\_authority & string & The organization that provides the initial id (see above) for the dataset. The naming authority should be uniquely specified by this attribute via reverse-DNS naming convention. & ACDD \\ \hline

\rowcolor{LightCyan} netcdf\_version\_id  & string & Version of netCDF libraries used to create this file. For example, "4.1.1" & GDS \\ \hline

\rowcolor{LightCyan} northernmost\_latitude & float & Decimal degrees north, range -90 to +90. This is equivalent to ACDD geospatial\_lat\_max. & GDS \\ \hline

\rowcolor{LightCyan} processing\_level & string & A textual description of the processing (or quality control) level of the data. & ACDD \& GDS \\ \hline

\rowcolor{LightCyan} product\_version & string & The product version of this data file & GDS \\ \hline

\rowcolor{LightCyan} project & string & The name of the project(s) principally responsible for originating this data. & ACDD \\ \hline

\rowcolor{LightCyan} publisher\_email & string & The email address of the person (or other entity specified by the publisher\_type attribute) responsible for publishing the data file or product to users, with its current metadata and format. & ACDD \\ \hline

\rowcolor{LightCyan} publisher\_name & string & The name of the person (or other entity specified by the publisher\_type attribute) responsible for publishing the data file or product to users, with its current metadata and format. & ACDD \\ \hline

\rowcolor{LightCyan} publisher\_url & string & The URL of the person (or other entity specified by the publisher\_type attribute) responsible for publishing the data file or product to users, with its current metadata and format. & ACDD \\ \hline

\rowcolor{LightCyan} references & string & Published or web-based references that describe the data or methods used to produce it. Recommend URIs (such as a URL or DOI) for papers or other references. This attribute is defined in the CF conventions. & ACDD \\ \hline

\rowcolor{LightCyan} source & string & Method of production of the original data. & CF \\ \hline

\rowcolor{LightCyan} sourthernmost\_latitude & float & Decimal degrees north, range -90 to +90. This is equivalent to ACDD geospatial\_lat\_min. & GDS \\ \hline

\rowcolor{LightCyan} spatial\_resolution & string & A string describing the approximate resolution of the product. & GDS \\ \hline

\rowcolor{LightCyan} standard\_name\_vocabulary & string & The name and version of the controlled vocabulary from which variable standard names are taken. & ACDD \\ \hline

\rowcolor{LightCyan} start\_time & string & Representative date and time of the end of the granule in the ISO 8601 compliant format of "yyyymmddThhmmssZ". & GDS \\ \hline

\rowcolor{LightCyan} stop\_time & string & Representative date and time of the end of the granule in the ISO 8601 compliant format of "yyyymmddThhmmssZ". & GDS \\ \hline

\rowcolor{LightCyan} summary & string & A paragraph describing the dataset, analogous to an abstract for a paper. & ACDD \\ \hline

\rowcolor{LightCyan} time\_coverage\_end & string & Identical to stop\_time. Included for increased ACDD compliance. & ACDD \\ \hline

\rowcolor{LightCyan} time\_coverage\_start & string & Identical to start\_time. Included for increased ACDD compliance. & ACDD \\ \hline

\rowcolor{LightCyan} title & string & A short phrase or sentence describing the dataset. In many discovery systems, the title will be displayed in the results list from a search, and therefore should be human readable and reasonable to display in a list of such names. This attribute is recommended by the NetCDF Users Guide (NUG) and the CF conventions. & CF, ACDD \\ \hline

\rowcolor{LightCyan} uuid & string & A Universally Unique Identifier (UUID). Numerous, simple tools can be used to create a UUID, which is inserted as the value of this attribute. See http://en.wikipedia.org/wiki/Universally\_Unique\_Identifier for more information and tools. & GDS \\ \hline

\rowcolor{LightCyan} westernmost\_longitude & float & Decimal degrees east, range -180 to +180. This is equivalent to ACDD geospatial\_lon\_min. & GDS \\ \hline

\end{longtable} % inserting contents from a python generated tex file


\subsection{GDS 2.0 netCDF Variable Attributes}
% Table 8-2 Variable attributes for GDS 2.0 netCDF data files
% \begin{longtable}{|p{0.15\textwidth}|p{0.25\textwidth}|p{0.5\textwidth}|p{0.1\textwidth}|}

% \caption{Table 8-2. Variable attributes for GDS 2.0 netCDF data files} \label{tab:variable-attributes} \\
% \hline \endhead
% \hline \endfoot
% \rowcolor{lightgray} \textbf{Variable Attribute Name} & \textbf{Format} & \textbf{Description} & \textbf{Source} \\ \hline
%     % Table 8-2 Variable attributes for GDS 2.0 netCDF data files
\begin{longtable}{|p{0.168\textwidth}|p{0.20\textwidth}|p{0.46\textwidth}|p{0.092\textwidth}|}
\caption{Table 8-2. Variable attributes for GDS 2.0 netCDF data files}
\label{tab:variable-attributes} \\ 
\hline \endhead
\hline \endfoot
\rowcolor{lightgray} \textbf{Variable Attribute Name} & \textbf{Format} & \textbf{Description} & \textbf{Source} \\ \hline
\rowcolor{LightCyan} \_FillValue & Must be the same as the variable type & A value used to indicate array elements containing no valid data. This value must be of the same type as the storage (packed) type; should be set as the minimum value for this type. Note that some netCDF readers are unable to cope with signed bytes and may, in these cases, report fill as 128. Some cases will be reported as unsigned bytes 0 to 255. Required for the majority of variables except mask and l2p\_flags. & CF \\ \hline

\rowcolor{LightCyan} units & string & Text description of the units, preferably S.I., and must be compatible with the Unidata UDUNITS-2 package [AD-5]. For a given variable (e.g. wind speed), these must be the same for each dataset. Required for the majority of variables except mask, quality\_level, and l2p\_flags. & CF, ACDD \\ \hline

\rowcolor{LightCyan} scale\_factor & Must be expressed in the unpacked data type & To be multiplied by the variable to recover the
original value. Defined by the producing
RDAC. Valid values within \texttt\{value\_min\} and
\texttt\{valid\_max\} should be transformed by
\texttt\{scale\_factor\} and \texttt\{add\_offset\}, otherwise
skipped to avoid floating point errors. & CF \\ \hline

\rowcolor{LightCyan} add\_offset & Must be expressed in the unpacked data type & To be added to the variable after multiplying by the scale factor to recover the original value. If only one of \texttt\{scale\_factor\} or \texttt\{add\_offset\} is needed, then both should be included anyway to avoid ambiguity, with \texttt\{scale\_factor\} defaulting to 1.0 and add\_offset defaulting to 0.0. Defined by the producing RDAC. & CF \\ \hline

\rowcolor{LightCyan} long\_name & string & A free-text descriptive variable name. & CF, ACDD \\ \hline

\rowcolor{LightCyan} valid\_min & Expressed in same data type as variable & Minimum valid value for this variable once they are packed (in storage type). The fill value should be outside this valid range. Note that some netCDF readers are unable to cope with signed bytes and may, in these cases, report valid min as 129. Some cases as unsigned bytes 0 to 255. Values outside of \texttt\{valid\_min\} and \texttt\{valid\_max\} will be treated as missing values. Required for all variables except variable time. & CF \\ \hline

\rowcolor{LightCyan} valid\_max & Expressed in same data type as variable & Maximum valid value for this variable once
they are packed (in storage type). The fill
value should be outside this valid range. Note
that some netCDF readers are unable to cope
with signed bytes and may, in these cases,
report valid min as 127. Required for all
variables except variable time. & CF \\ \hline

\rowcolor{LightCyan} standard\_name & string & Where defined, a standard and unique
description of a physical quantity. For the
complete list of standard name strings, see
[AD-8]. \textbf\{Do not\} include this attribute if no
\texttt\{standard\_name\} exists. & CF, ACDD \\ \hline

\rowcolor{LightCyan} comment & string & Miscellaneous information about the variable or the methods used to produce it. & CF \\ \hline

\rowcolor{LightCyan} source & string & \textbf\{For L2P and L3 files\}: For a data variable with
a single source, use the GHRSST unique
string listed in Table 7-10 if the source is a
GHRSST SST product. For other sources,
following the best practice described in
Section 7.9 to create the character string.

If the data variable contains multiple sources,
set this string to be the relevant “sources of”
variable name. For example, if multiple wind
speed sources are used, set \texttt\{source =\}
sources\_of\_wind\_speed.

\textbf\{For L4 and GMPE files\}: follow the \texttt\{source\}
convention used for the global attribute of the
same name, but provide in the commaseparated list only the sources relevant to this
variable. & CF \\ \hline

\rowcolor{LightCyan} references & string & Published or web-based references that describe the data or methods used to produce it. Note that while at least one reference is required in the global attributes (See Table 8-1), references to this specific data variable may also be given. & CF \\ \hline

\rowcolor{LightCyan} axis & String & For use with coordinate variables only. The attribute 'axis' may be attached to a coordinate variable and given one of the values “X”, “Y”, “Z”, or “T”, which stand for a longitude, latitude, vertical, or time axis respectively. See: \url{http://cfpcmdi.llnl.gov/documents/cfconventions/1.4/cfconventions.html#coordinate-types} & CF \\ \hline

\rowcolor{LightCyan} positive & String & For use with a vertical coordinate variables
only. May have the value “up” or “down”. For
example, if an oceanographic netCDF file
encodes the depth of the surface as 0 and the
depth of 1000 meters as 1000 then the axis
would set positive to “down”. If a depth of
1000 meters was encoded as -1000, then
positive would be set to “up”. See the section
on vertical-coordinate in [AD-3] & CF \\ \hline

\rowcolor{LightCyan} coordinates & String & Identifies auxiliary coordinate variables, label variables, and alternate coordinate variables. See the section on coordinate-system in [AD3]. This attribute must be provided if the data are on a non-regular lat/lon grid (map projection or swath data). & CF \\ \hline

\rowcolor{LightCyan} grid\_mapping & String & Use this for data variables that are on a projected grid. The attribute takes a string value that is the name of another variable in the file that provides the description of the mapping via a collection of attached attributes. That named variable is called a grid mapping variable and is of arbitrary type since it contains no data. Its purpose is to act as a container for the attributes that define the mapping. See the section on mappings-andprojections in [AD-3] & CF \\ \hline

\rowcolor{LightCyan} flag\_mappings & String & Space-separated list of text descriptions associated in strict order with conditions set by either flag\_values or flag\_masks. Words within a phrase should be connected with underscores. & CF \\ \hline

\rowcolor{LightCyan} flag\_values & Must be the same as
the variable type & Comma-separated array of valid, mutually exclusive variable values (required when the bit field contains enumerated values; i.e., a “list” of conditions). Used primarily for \texttt\{quality\_level\} and “\texttt\{sources\_of\_xxx\}” variables. & CF \\ \hline

\rowcolor{LightCyan} flag\_masks & Must be the same as the variable type & Comma-separated array of valid variable
masks (required when the bit field contains
independent Boolean conditions; i.e., a bit
“mask”). Used primarily for \texttt\{l2p\_flags\}
variable.

\emph\{Note: CF allows the use of both flag\_masks
and flag\_values attributes in a single variable
to create sets of masks that each have their
own list of flag\_values (see \url{http://cfpcmdi.llnl.gov/documents/cfconventions/1.5/ch03s05.html#id2710752} for
examples), but this practice is discouraged.\} & CF \\ \hline

\rowcolor{LightCyan} depth & String & Use this to indicate the depth for which the
SST data are valid. & GDS \\ \hline

\rowcolor{LightCyan} height & String & Use this to indicate the height for which the wind data are specified. & GDS \\ \hline

\rowcolor{LightCyan} time\_offset & Must be expressed in
the unpacked data
type & Difference in hours between an ancillary field such as \texttt\{wind\_speed\} and the SST observation time & GDS \\ \hline

\end{longtable} % inserting contents from a python generated tex file
% \end{longtable}
% Table 8-2 Variable attributes for GDS 2.0 netCDF data files
\begin{longtable}{|p{0.168\textwidth}|p{0.20\textwidth}|p{0.46\textwidth}|p{0.092\textwidth}|}
\caption{Table 8-2. Variable attributes for GDS 2.0 netCDF data files}
\label{tab:variable-attributes} \\ 
\hline \endhead
\hline \endfoot
\rowcolor{lightgray} \textbf{Variable Attribute Name} & \textbf{Format} & \textbf{Description} & \textbf{Source} \\ \hline
\rowcolor{LightCyan} \_FillValue & Must be the same as the variable type & A value used to indicate array elements containing no valid data. This value must be of the same type as the storage (packed) type; should be set as the minimum value for this type. Note that some netCDF readers are unable to cope with signed bytes and may, in these cases, report fill as 128. Some cases will be reported as unsigned bytes 0 to 255. Required for the majority of variables except mask and l2p\_flags. & CF \\ \hline

\rowcolor{LightCyan} units & string & Text description of the units, preferably S.I., and must be compatible with the Unidata UDUNITS-2 package [AD-5]. For a given variable (e.g. wind speed), these must be the same for each dataset. Required for the majority of variables except mask, quality\_level, and l2p\_flags. & CF, ACDD \\ \hline

\rowcolor{LightCyan} scale\_factor & Must be expressed in the unpacked data type & To be multiplied by the variable to recover the
original value. Defined by the producing
RDAC. Valid values within \texttt\{value\_min\} and
\texttt\{valid\_max\} should be transformed by
\texttt\{scale\_factor\} and \texttt\{add\_offset\}, otherwise
skipped to avoid floating point errors. & CF \\ \hline

\rowcolor{LightCyan} add\_offset & Must be expressed in the unpacked data type & To be added to the variable after multiplying by the scale factor to recover the original value. If only one of \texttt\{scale\_factor\} or \texttt\{add\_offset\} is needed, then both should be included anyway to avoid ambiguity, with \texttt\{scale\_factor\} defaulting to 1.0 and add\_offset defaulting to 0.0. Defined by the producing RDAC. & CF \\ \hline

\rowcolor{LightCyan} long\_name & string & A free-text descriptive variable name. & CF, ACDD \\ \hline

\rowcolor{LightCyan} valid\_min & Expressed in same data type as variable & Minimum valid value for this variable once they are packed (in storage type). The fill value should be outside this valid range. Note that some netCDF readers are unable to cope with signed bytes and may, in these cases, report valid min as 129. Some cases as unsigned bytes 0 to 255. Values outside of \texttt\{valid\_min\} and \texttt\{valid\_max\} will be treated as missing values. Required for all variables except variable time. & CF \\ \hline

\rowcolor{LightCyan} valid\_max & Expressed in same data type as variable & Maximum valid value for this variable once
they are packed (in storage type). The fill
value should be outside this valid range. Note
that some netCDF readers are unable to cope
with signed bytes and may, in these cases,
report valid min as 127. Required for all
variables except variable time. & CF \\ \hline

\rowcolor{LightCyan} standard\_name & string & Where defined, a standard and unique
description of a physical quantity. For the
complete list of standard name strings, see
[AD-8]. \textbf\{Do not\} include this attribute if no
\texttt\{standard\_name\} exists. & CF, ACDD \\ \hline

\rowcolor{LightCyan} comment & string & Miscellaneous information about the variable or the methods used to produce it. & CF \\ \hline

\rowcolor{LightCyan} source & string & \textbf\{For L2P and L3 files\}: For a data variable with
a single source, use the GHRSST unique
string listed in Table 7-10 if the source is a
GHRSST SST product. For other sources,
following the best practice described in
Section 7.9 to create the character string.

If the data variable contains multiple sources,
set this string to be the relevant “sources of”
variable name. For example, if multiple wind
speed sources are used, set \texttt\{source =\}
sources\_of\_wind\_speed.

\textbf\{For L4 and GMPE files\}: follow the \texttt\{source\}
convention used for the global attribute of the
same name, but provide in the commaseparated list only the sources relevant to this
variable. & CF \\ \hline

\rowcolor{LightCyan} references & string & Published or web-based references that describe the data or methods used to produce it. Note that while at least one reference is required in the global attributes (See Table 8-1), references to this specific data variable may also be given. & CF \\ \hline

\rowcolor{LightCyan} axis & String & For use with coordinate variables only. The attribute 'axis' may be attached to a coordinate variable and given one of the values “X”, “Y”, “Z”, or “T”, which stand for a longitude, latitude, vertical, or time axis respectively. See: \url{http://cfpcmdi.llnl.gov/documents/cfconventions/1.4/cfconventions.html#coordinate-types} & CF \\ \hline

\rowcolor{LightCyan} positive & String & For use with a vertical coordinate variables
only. May have the value “up” or “down”. For
example, if an oceanographic netCDF file
encodes the depth of the surface as 0 and the
depth of 1000 meters as 1000 then the axis
would set positive to “down”. If a depth of
1000 meters was encoded as -1000, then
positive would be set to “up”. See the section
on vertical-coordinate in [AD-3] & CF \\ \hline

\rowcolor{LightCyan} coordinates & String & Identifies auxiliary coordinate variables, label variables, and alternate coordinate variables. See the section on coordinate-system in [AD3]. This attribute must be provided if the data are on a non-regular lat/lon grid (map projection or swath data). & CF \\ \hline

\rowcolor{LightCyan} grid\_mapping & String & Use this for data variables that are on a projected grid. The attribute takes a string value that is the name of another variable in the file that provides the description of the mapping via a collection of attached attributes. That named variable is called a grid mapping variable and is of arbitrary type since it contains no data. Its purpose is to act as a container for the attributes that define the mapping. See the section on mappings-andprojections in [AD-3] & CF \\ \hline

\rowcolor{LightCyan} flag\_mappings & String & Space-separated list of text descriptions associated in strict order with conditions set by either flag\_values or flag\_masks. Words within a phrase should be connected with underscores. & CF \\ \hline

\rowcolor{LightCyan} flag\_values & Must be the same as
the variable type & Comma-separated array of valid, mutually exclusive variable values (required when the bit field contains enumerated values; i.e., a “list” of conditions). Used primarily for \texttt\{quality\_level\} and “\texttt\{sources\_of\_xxx\}” variables. & CF \\ \hline

\rowcolor{LightCyan} flag\_masks & Must be the same as the variable type & Comma-separated array of valid variable
masks (required when the bit field contains
independent Boolean conditions; i.e., a bit
“mask”). Used primarily for \texttt\{l2p\_flags\}
variable.

\emph\{Note: CF allows the use of both flag\_masks
and flag\_values attributes in a single variable
to create sets of masks that each have their
own list of flag\_values (see \url{http://cfpcmdi.llnl.gov/documents/cfconventions/1.5/ch03s05.html#id2710752} for
examples), but this practice is discouraged.\} & CF \\ \hline

\rowcolor{LightCyan} depth & String & Use this to indicate the depth for which the
SST data are valid. & GDS \\ \hline

\rowcolor{LightCyan} height & String & Use this to indicate the height for which the wind data are specified. & GDS \\ \hline

\rowcolor{LightCyan} time\_offset & Must be expressed in
the unpacked data
type & Difference in hours between an ancillary field such as \texttt\{wind\_speed\} and the SST observation time & GDS \\ \hline

\end{longtable} % inserting contents from a python generated tex file

\subsection{GDS 2.0 coordinate variable definitions}
NetCDF coordinate variables provide scales for the space and time axes for the multidimensional data
arrays, and must be included for all dimensions that can be identified as spatio-temporal axes.
Coordinate arrays are used to geolocate data arrays on non-orthogonal grids, such as images in the
original pixel/scan line space, or complicated map projections. Required attributes are \texttt{units} and
\texttt{\_FillValue}. Elements of the coordinate array need not be monotonically ordered. The data type
can be any and scaling may be implemented if required. \texttt{add\_offset} and \texttt{scale\_factor} have to
be adjusted according to the sensor resolution and the product spatial coverage. If the packed values
can not stand on a short, float can be used instead (multiplying the size of these variables by two).
\par \vspace{0.1in}

'\texttt{time}' is the reference time of the SST data array. The GDS 2.0 specifies that this reference time
should be extracted or computed to the nearest second and then coded as continuous UTC time
coordinates in \textbf{seconds from 00:00:00 UTC January 1, 1981} (which is the definition of the \textbf{GHRSST}
origin time, chosen to approximate the start of useful AVHRR SST data record). Note that the use of
UDUNITS in GHRSST implies that that calendar to be used is the default mixed Gregorian/Julian
calendar.
\par \vspace{0.1in}

The reference time used is dependent on the <Processing Level> of the data and is defined as
follows:
\par \vspace{0.1in}
\begin{itemize}
    \item L2P: start time of granule;
    \item L3U: start time of granule;
    \item L3C and L3S: centre time of the collation window;
    \item L4 and GMPE: nominal time of the analysis
\end{itemize}
\par \vspace{0.1in}

The coordinate variable '\texttt{time}' is intended to minimize the size of the \texttt{sst\_dtime} variable (e.g., see
Section 9.4), which stores offsets from the reference time in seconds for each SST pixel. \texttt{'time'} also
facilitates aggregation of all files of a given dataset along the time axis with such tools as THREDDS
and LAS.
\par \vspace{0.1in}

x (columns) and y (lines) grid dimensions are referred either as \texttt{'lat'} and \texttt{'lon'} or as \texttt{'ni'} and \texttt{'nj'}.
\texttt{lon} and \texttt{lat} must be used if data are mapped on a regular grid (some geostationary products). \texttt{ni}
and \texttt{nj} are used if data are mapped on a non-regular grid (curvilinear coordinates) or following the 
sensor scanning pattern (scan line, swath). It is preferred that \texttt{ni} should be used for the across-track
dimension and \texttt{nj} for the along-track dimension.
\par \vspace{0.1in}

Coordinate vectors are used for data arrays located on orthogonal (but not necessarily regularly
spaced) grids, such as a geographic (lat-lon) map projections. The only required attribute is \texttt{units}.
The elements of a coordinate vector array should be in monotonically increasing or decreasing order.
The data type can be any and scaling may be implemented if required.
\par \vspace{0.1in}

A \texttt{coordinate's} variable (= \texttt{"lon lat"}): must be provided if the data are on a non-regular lat/lon
grid (map projection or swath data).
\par \vspace{0.1in}

A \texttt{grid\_mapping} (= "projection name"): must be provided if the data are mapped following a
projection. Refer to the CF convention [AD-3] for standard projection names. 
\par
\pagebreak

\subsubsection{Native datasets}
Hoc est casus simplex. Multae L3, L4, et GMPE comoediae, necnon quaedam geostationaria L2P comoediae, in ordinaria lat/lon tabula praebentur. In huiusmodi projectione, solum duo coordinate sunt requisitae et vectorum formis servari possunt. Longitudines debent variare ab -180 ad +180, id est ab 180 gradibus Occidentem ad 180 gradibus Orientem. Latitudines debent variare ab -90 ad +90, id est ab 90 gradibus Meridiem ad 90 gradibus Septentrionem. Non debet esse \_FillValue pro latitudine et longitudine, et omnes SST pixeles debent habere validum latitudinis et longitudinis valorem.
\par \vspace{0.1in}

Recommendatur ut tempus dimensionem pro Level 3 et Level 4 data prodigia ut infinita specificetur. Nota quod tempus dimensio pro L2P data est stricta definita ut tempus=1 (infinita dimensio non permittitur). Hoc strictum definitum est quia L2P data sunt swath based et geospatial informatio potest mutare per consecutive tempus slabs.
\par \vspace{0.1in}
In GHRSST L3 et L4 granulis, solum unum tempus dimensio (tempus=1) est, et variabilis tempus solum unum valorem habet (secunda post 1981), sed infinitum tempus dimensionem permittit netCDF instrumenta et utilitates facile concatenare (et exempli gratia, mediare) seriem de tempore consecutive GHRSST granulis. Sequens CDL exemplum dat:
\par \vspace{0.1in}

\begin{verbatim}
    netcdf example {
        dimensions:
        lat = 1801 ;
        lon = 3600 ;
        time = UNLIMITED ; // (strictly set to 1 for L2P)
        variables:
        …
    }
\end{verbatim}
\par \vspace{0.1in}
Pro his casibus, dimensiones et coordinae variabiles debent uti pro regulari lat/lon tabula, ut in Tabula 8-3 monstratur. Nullae specificae variabiles attributi sunt requisitae pro aliis variabilibus (ut sea\_surface\_temperature, ut in exemplo dat in Tabula 8-3).
\par \vspace{0.1in}

% Table 8-3. Example of a native dataset
\begin{longtable}{|p{\textwidth}|}
\caption{Example CDL description of native dataset}
\label{tab:cdl-native} \\
\hline \endhead
\hline \endfoot
netcdf native example\\
dimensions\\
\hline
\rowcolor{YellowGreen}  i = 90\\
\rowcolor{YellowGreen}  i\_g = 90\\
\rowcolor{YellowGreen}  j = 90\\
\rowcolor{YellowGreen}  j\_g = 90\\
\rowcolor{YellowGreen}  k = 50\\
\rowcolor{YellowGreen}  k\_u = 50\\
\rowcolor{YellowGreen}  k\_l = 50\\
\rowcolor{YellowGreen}  k\_p1 = 51\\
\rowcolor{YellowGreen}  tile = 13\\
\rowcolor{YellowGreen}  nb = 4\\
\rowcolor{YellowGreen}  nv = 2\\
\hline

coordinates\\
\hline
\rowcolor{Apricot}\hspace{0.5cm}int32 i (i)\\
\rowcolor{Apricot}\hspace{0.5cm}\hspace{0.5cm}i:axis = "X"\\
\rowcolor{Apricot}\hspace{0.5cm}\hspace{0.5cm}i:long\_name = "grid index in x for variables at tracer and 'v' locations"\\
\rowcolor{Apricot}\hspace{0.5cm}\hspace{0.5cm}i:swap\_dim = "XC"\\
\rowcolor{Apricot}\hspace{0.5cm}\hspace{0.5cm}i:comment = "In the Arakawa C-grid system, tracer (e.g., THETA) and 'v' variables (e.g., VVEL) have the same x coordinate on the model grid."\\
\rowcolor{Apricot}\hspace{0.5cm}\hspace{0.5cm}i:coverage\_content\_type = "coordinate"\\
\rowcolor{Apricot}\hspace{0.5cm}int32 i\_g (i\_g)\\
\rowcolor{Apricot}\hspace{0.5cm}\hspace{0.5cm}i\_g:axis = "X"\\
\rowcolor{Apricot}\hspace{0.5cm}\hspace{0.5cm}i\_g:long\_name = "grid index in x for variables at 'u' and 'g' locations"\\
\rowcolor{Apricot}\hspace{0.5cm}\hspace{0.5cm}i\_g:c\_grid\_axis\_shift = "-0.5"\\
\rowcolor{Apricot}\hspace{0.5cm}\hspace{0.5cm}i\_g:swap\_dim = "XG"\\
\rowcolor{Apricot}\hspace{0.5cm}\hspace{0.5cm}i\_g:comment = "In the Arakawa C-grid system, 'u' (e.g., UVEL) and 'g' variables (e.g., XG) have the same x coordinate on the model grid."\\
\rowcolor{Apricot}\hspace{0.5cm}\hspace{0.5cm}i\_g:coverage\_content\_type = "coordinate"\\
\rowcolor{Apricot}\hspace{0.5cm}int32 j (j)\\
\rowcolor{Apricot}\hspace{0.5cm}\hspace{0.5cm}j:axis = "Y"\\
\rowcolor{Apricot}\hspace{0.5cm}\hspace{0.5cm}j:long\_name = "grid index in y for variables at tracer and 'u' locations"\\
\rowcolor{Apricot}\hspace{0.5cm}\hspace{0.5cm}j:swap\_dim = "YC"\\
\rowcolor{Apricot}\hspace{0.5cm}\hspace{0.5cm}j:comment = "In the Arakawa C-grid system, tracer (e.g., THETA) and 'u' variables (e.g., UVEL) have the same y coordinate on the model grid."\\
\rowcolor{Apricot}\hspace{0.5cm}\hspace{0.5cm}j:coverage\_content\_type = "coordinate"\\
\rowcolor{Apricot}\hspace{0.5cm}int32 j\_g (j\_g)\\
\rowcolor{Apricot}\hspace{0.5cm}\hspace{0.5cm}j\_g:axis = "Y"\\
\rowcolor{Apricot}\hspace{0.5cm}\hspace{0.5cm}j\_g:long\_name = "grid index in y for variables at 'v' and 'g' locations"\\
\rowcolor{Apricot}\hspace{0.5cm}\hspace{0.5cm}j\_g:c\_grid\_axis\_shift = "-0.5"\\
\rowcolor{Apricot}\hspace{0.5cm}\hspace{0.5cm}j\_g:swap\_dim = "YG"\\
\rowcolor{Apricot}\hspace{0.5cm}\hspace{0.5cm}j\_g:comment = "In the Arakawa C-grid system, 'v' (e.g., VVEL) and 'g' variables (e.g., XG) have the same y coordinate."\\
\rowcolor{Apricot}\hspace{0.5cm}\hspace{0.5cm}j\_g:coverage\_content\_type = "coordinate"\\
\rowcolor{Apricot}\hspace{0.5cm}int32 k (k)\\
\rowcolor{Apricot}\hspace{0.5cm}\hspace{0.5cm}k:axis = "Z"\\
\rowcolor{Apricot}\hspace{0.5cm}\hspace{0.5cm}k:long\_name = "grid index in z for tracer variables"\\
\rowcolor{Apricot}\hspace{0.5cm}\hspace{0.5cm}k:swap\_dim = "Z"\\
\rowcolor{Apricot}\hspace{0.5cm}\hspace{0.5cm}k:coverage\_content\_type = "coordinate"\\
\rowcolor{Apricot}\hspace{0.5cm}int32 k\_u (k\_u)\\
\rowcolor{Apricot}\hspace{0.5cm}\hspace{0.5cm}k\_u:axis = "Z"\\
\rowcolor{Apricot}\hspace{0.5cm}\hspace{0.5cm}k\_u:long\_name = "grid index in z corresponding to the bottom face of tracer grid cells ('w' locations)"\\
\rowcolor{Apricot}\hspace{0.5cm}\hspace{0.5cm}k\_u:c\_grid\_axis\_shift = "0.5"\\
\rowcolor{Apricot}\hspace{0.5cm}\hspace{0.5cm}k\_u:swap\_dim = "Zu"\\
\rowcolor{Apricot}\hspace{0.5cm}\hspace{0.5cm}k\_u:comment = "First index corresponds to the bottom face of the uppermost tracer grid cell. The use of 'u' in the variable name follows the MITgcm convention for naming the bottom face of ocean tracer grid cells."\\
\rowcolor{Apricot}\hspace{0.5cm}\hspace{0.5cm}k\_u:coverage\_content\_type = "coordinate"\\
\rowcolor{Apricot}\hspace{0.5cm}int32 k\_l (k\_l)\\
\rowcolor{Apricot}\hspace{0.5cm}\hspace{0.5cm}k\_l:axis = "Z"\\
\rowcolor{Apricot}\hspace{0.5cm}\hspace{0.5cm}k\_l:long\_name = "grid index in z corresponding to the top face of tracer grid cells ('w' locations)"\\
\rowcolor{Apricot}\hspace{0.5cm}\hspace{0.5cm}k\_l:c\_grid\_axis\_shift = "-0.5"\\
\rowcolor{Apricot}\hspace{0.5cm}\hspace{0.5cm}k\_l:swap\_dim = "Zl"\\
\rowcolor{Apricot}\hspace{0.5cm}\hspace{0.5cm}k\_l:comment = "First index corresponds to the top face of the uppermost tracer grid cell. The use of 'l' in the variable name follows the MITgcm convention for naming the top face of ocean tracer grid cells."\\
\rowcolor{Apricot}\hspace{0.5cm}\hspace{0.5cm}k\_l:coverage\_content\_type = "coordinate"\\
\rowcolor{Apricot}\hspace{0.5cm}int32 k\_p1 (k\_p1)\\
\rowcolor{Apricot}\hspace{0.5cm}\hspace{0.5cm}k\_p1:axis = "Z"\\
\rowcolor{Apricot}\hspace{0.5cm}\hspace{0.5cm}k\_p1:long\_name = "grid index in z for variables at 'w' locations"\\
\rowcolor{Apricot}\hspace{0.5cm}\hspace{0.5cm}k\_p1:c\_grid\_axis\_shift = "[-0.5  0.5]"\\
\rowcolor{Apricot}\hspace{0.5cm}\hspace{0.5cm}k\_p1:swap\_dim = "Zp1"\\
\rowcolor{Apricot}\hspace{0.5cm}\hspace{0.5cm}k\_p1:comment = "Includes top of uppermost model tracer cell (k\_p1=0) and bottom of lowermost tracer cell (k\_p1=51)."\\
\rowcolor{Apricot}\hspace{0.5cm}\hspace{0.5cm}k\_p1:coverage\_content\_type = "coordinate"\\
\rowcolor{Apricot}\hspace{0.5cm}int32 tile (tile)\\
\rowcolor{Apricot}\hspace{0.5cm}\hspace{0.5cm}tile:long\_name = "lat-lon-cap tile index"\\
\rowcolor{Apricot}\hspace{0.5cm}\hspace{0.5cm}tile:comment = "The ECCO V4 horizontal model grid is divided into 13 tiles of 90x90 cells for convenience."\\
\rowcolor{Apricot}\hspace{0.5cm}\hspace{0.5cm}tile:coverage\_content\_type = "coordinate"\\
\rowcolor{Apricot}\hspace{0.5cm}float32 XC (tile, j, i)\\
\rowcolor{Apricot}\hspace{0.5cm}\hspace{0.5cm}XC:long\_name = "longitude of tracer grid cell center"\\
\rowcolor{Apricot}\hspace{0.5cm}\hspace{0.5cm}XC:units = "degrees\_east"\\
\rowcolor{Apricot}\hspace{0.5cm}\hspace{0.5cm}XC:coordinate = "YC XC"\\
\rowcolor{Apricot}\hspace{0.5cm}\hspace{0.5cm}XC:bounds = "XC\_bnds"\\
\rowcolor{Apricot}\hspace{0.5cm}\hspace{0.5cm}XC:comment = "nonuniform grid spacing"\\
\rowcolor{Apricot}\hspace{0.5cm}\hspace{0.5cm}XC:coverage\_content\_type = "coordinate"\\
\rowcolor{Apricot}\hspace{0.5cm}\hspace{0.5cm}XC:standard\_name = "longitude"\\
\rowcolor{Apricot}\hspace{0.5cm}float32 YC (tile, j, i)\\
\rowcolor{Apricot}\hspace{0.5cm}\hspace{0.5cm}YC:long\_name = "latitude of tracer grid cell center"\\
\rowcolor{Apricot}\hspace{0.5cm}\hspace{0.5cm}YC:units = "degrees\_north"\\
\rowcolor{Apricot}\hspace{0.5cm}\hspace{0.5cm}YC:coordinate = "YC XC"\\
\rowcolor{Apricot}\hspace{0.5cm}\hspace{0.5cm}YC:bounds = "YC\_bnds"\\
\rowcolor{Apricot}\hspace{0.5cm}\hspace{0.5cm}YC:comment = "nonuniform grid spacing"\\
\rowcolor{Apricot}\hspace{0.5cm}\hspace{0.5cm}YC:coverage\_content\_type = "coordinate"\\
\rowcolor{Apricot}\hspace{0.5cm}\hspace{0.5cm}YC:standard\_name = "latitude"\\
\rowcolor{Apricot}\hspace{0.5cm}float32 XG (tile, j\_g, i\_g)\\
\rowcolor{Apricot}\hspace{0.5cm}\hspace{0.5cm}XG:long\_name = "longitude of 'southwest' corner of tracer grid cell"\\
\rowcolor{Apricot}\hspace{0.5cm}\hspace{0.5cm}XG:units = "degrees\_east"\\
\rowcolor{Apricot}\hspace{0.5cm}\hspace{0.5cm}XG:coordinate = "YG XG"\\
\rowcolor{Apricot}\hspace{0.5cm}\hspace{0.5cm}XG:comment = "Nonuniform grid spacing. Note: 'southwest' does not correspond to geographic orientation but is used for convenience to describe the computational grid. See MITgcm dcoumentation for details."\\
\rowcolor{Apricot}\hspace{0.5cm}\hspace{0.5cm}XG:coverage\_content\_type = "coordinate"\\
\rowcolor{Apricot}\hspace{0.5cm}\hspace{0.5cm}XG:standard\_name = "longitude"\\
\rowcolor{Apricot}\hspace{0.5cm}float32 YG (tile, j\_g, i\_g)\\
\rowcolor{Apricot}\hspace{0.5cm}\hspace{0.5cm}YG:long\_name = "latitude of 'southwest' corner of tracer grid cell"\\
\rowcolor{Apricot}\hspace{0.5cm}\hspace{0.5cm}YG:units = "degrees\_north"\\
\rowcolor{Apricot}\hspace{0.5cm}\hspace{0.5cm}YG:comment = "Nonuniform grid spacing. Note: 'southwest' does not correspond to geographic orientation but is used for convenience to describe the computational grid. See MITgcm dcoumentation for details."\\
\rowcolor{Apricot}\hspace{0.5cm}\hspace{0.5cm}YG:coverage\_content\_type = "coordinate"\\
\rowcolor{Apricot}\hspace{0.5cm}\hspace{0.5cm}YG:standard\_name = "latitude"\\
\rowcolor{Apricot}\hspace{0.5cm}\hspace{0.5cm}YG:coordinates = "YG XG"\\
\rowcolor{Apricot}\hspace{0.5cm}float32 Z (k)\\
\rowcolor{Apricot}\hspace{0.5cm}\hspace{0.5cm}Z:long\_name = "depth of tracer grid cell center"\\
\rowcolor{Apricot}\hspace{0.5cm}\hspace{0.5cm}Z:units = "m"\\
\rowcolor{Apricot}\hspace{0.5cm}\hspace{0.5cm}Z:positive = "up"\\
\rowcolor{Apricot}\hspace{0.5cm}\hspace{0.5cm}Z:bounds = "Z\_bnds"\\
\rowcolor{Apricot}\hspace{0.5cm}\hspace{0.5cm}Z:comment = "Non-uniform vertical spacing."\\
\rowcolor{Apricot}\hspace{0.5cm}\hspace{0.5cm}Z:coverage\_content\_type = "coordinate"\\
\rowcolor{Apricot}\hspace{0.5cm}\hspace{0.5cm}Z:standard\_name = "depth"\\
\rowcolor{Apricot}\hspace{0.5cm}float32 Zp1 (k\_p1)\\
\rowcolor{Apricot}\hspace{0.5cm}\hspace{0.5cm}Zp1:long\_name = "depth of top/bottom face of tracer grid cell"\\
\rowcolor{Apricot}\hspace{0.5cm}\hspace{0.5cm}Zp1:units = "m"\\
\rowcolor{Apricot}\hspace{0.5cm}\hspace{0.5cm}Zp1:positive = "up"\\
\rowcolor{Apricot}\hspace{0.5cm}\hspace{0.5cm}Zp1:comment = "Contains one element more than the number of vertical layers. First element is 0m, the depth of the top face of the uppermost grid cell. Last element is the depth of the bottom face of the deepest grid cell."\\
\rowcolor{Apricot}\hspace{0.5cm}\hspace{0.5cm}Zp1:coverage\_content\_type = "coordinate"\\
\rowcolor{Apricot}\hspace{0.5cm}\hspace{0.5cm}Zp1:standard\_name = "depth"\\
\rowcolor{Apricot}\hspace{0.5cm}float32 Zu (k\_u)\\
\rowcolor{Apricot}\hspace{0.5cm}\hspace{0.5cm}Zu:long\_name = "depth of bottom face of tracer grid cell"\\
\rowcolor{Apricot}\hspace{0.5cm}\hspace{0.5cm}Zu:units = "m"\\
\rowcolor{Apricot}\hspace{0.5cm}\hspace{0.5cm}Zu:positive = "up"\\
\rowcolor{Apricot}\hspace{0.5cm}\hspace{0.5cm}Zu:comment = "First element is -10m, the depth of the bottom face of the uppermost tracer grid cell. Last element is the depth of the bottom face of the deepest grid cell. The use of 'u' in the variable name follows the MITgcm convention for naming the bottom face of ocean tracer grid cells."\\
\rowcolor{Apricot}\hspace{0.5cm}\hspace{0.5cm}Zu:coverage\_content\_type = "coordinate"\\
\rowcolor{Apricot}\hspace{0.5cm}\hspace{0.5cm}Zu:standard\_name = "depth"\\
\rowcolor{Apricot}\hspace{0.5cm}float32 Zl (k\_l)\\
\rowcolor{Apricot}\hspace{0.5cm}\hspace{0.5cm}Zl:long\_name = "depth of top face of tracer grid cell"\\
\rowcolor{Apricot}\hspace{0.5cm}\hspace{0.5cm}Zl:units = "m"\\
\rowcolor{Apricot}\hspace{0.5cm}\hspace{0.5cm}Zl:positive = "up"\\
\rowcolor{Apricot}\hspace{0.5cm}\hspace{0.5cm}Zl:comment = "First element is 0m, the depth of the top face of the uppermost tracer grid cell (i.e., the ocean surface). Last element is the depth of the top face of the deepest grid cell. The use of 'l' in the variable name follows the MITgcm convention for naming the top face of ocean tracer grid cells."\\
\rowcolor{Apricot}\hspace{0.5cm}\hspace{0.5cm}Zl:coverage\_content\_type = "coordinate"\\
\rowcolor{Apricot}\hspace{0.5cm}\hspace{0.5cm}Zl:standard\_name = "depth"\\
\rowcolor{Apricot}\hspace{0.5cm}float32 XC\_bnds (tile, j, i, nb)\\
\rowcolor{Apricot}\hspace{0.5cm}\hspace{0.5cm}XC\_bnds:comment = "Bounds array follows CF conventions. XC\_bnds[i,j,0] = 'southwest' corner (j-1, i-1), XC\_bnds[i,j,1] = 'southeast' corner (j-1, i+1), XC\_bnds[i,j,2] = 'northeast' corner (j+1, i+1), XC\_bnds[i,j,3]  = 'northwest' corner (j+1, i-1). Note: 'southwest', 'southeast', northwest', and 'northeast' do not correspond to geographic orientation but are used for convenience to describe the computational grid. See MITgcm dcoumentation for details."\\
\rowcolor{Apricot}\hspace{0.5cm}\hspace{0.5cm}XC\_bnds:coverage\_content\_type = "coordinate"\\
\rowcolor{Apricot}\hspace{0.5cm}\hspace{0.5cm}XC\_bnds:long\_name = "longitudes of tracer grid cell corners"\\
\rowcolor{Apricot}\hspace{0.5cm}float32 YC\_bnds (tile, j, i, nb)\\
\rowcolor{Apricot}\hspace{0.5cm}\hspace{0.5cm}YC\_bnds:comment = "Bounds array follows CF conventions. YC\_bnds[i,j,0] = 'southwest' corner (j-1, i-1), YC\_bnds[i,j,1] = 'southeast' corner (j-1, i+1), YC\_bnds[i,j,2] = 'northeast' corner (j+1, i+1), YC\_bnds[i,j,3]  = 'northwest' corner (j+1, i-1). Note: 'southwest', 'southeast', northwest', and 'northeast' do not correspond to geographic orientation but are used for convenience to describe the computational grid. See MITgcm dcoumentation for details."\\
\rowcolor{Apricot}\hspace{0.5cm}\hspace{0.5cm}YC\_bnds:coverage\_content\_type = "coordinate"\\
\rowcolor{Apricot}\hspace{0.5cm}\hspace{0.5cm}YC\_bnds:long\_name = "latitudes of tracer grid cell corners"\\
\rowcolor{Apricot}\hspace{0.5cm}float32 Z\_bnds (k, nv)\\
\rowcolor{Apricot}\hspace{0.5cm}\hspace{0.5cm}Z\_bnds:comment = "One pair of depths for each vertical level."\\
\rowcolor{Apricot}\hspace{0.5cm}\hspace{0.5cm}Z\_bnds:coverage\_content\_type = "coordinate"\\
\rowcolor{Apricot}\hspace{0.5cm}\hspace{0.5cm}Z\_bnds:long\_name = "depths of top and bottom faces of tracer grid cell"\\
\hline

data variables\\
\hline
\hspace{0.5cm}float32 DIFFKR (k, tile, j, i)\\
\hspace{0.5cm}\hspace{0.5cm}DIFFKR:\_FillValue = "9.969209968386869e+36"\\
\hspace{0.5cm}\hspace{0.5cm}DIFFKR:coverage\_content\_type = "modelResult"\\
\hspace{0.5cm}\hspace{0.5cm}DIFFKR:long\_name = "Vertical diffusivity"\\
\hspace{0.5cm}\hspace{0.5cm}DIFFKR:units = "m2 s-1"\\
\hspace{0.5cm}\hspace{0.5cm}DIFFKR:comment = "Background vertical diffusion coefficient for temperature and salinity. Total vertical diffusivity includes background diffusivity plus contributions from the GGL90 vertical mixing and the Gent-McWilliams/Redi parameterizations. Note: DIFFKR is a model control variable and has been optimized from a spatially-invariant first-guess value of 1e-5 m2 s-1."\\
\hspace{0.5cm}\hspace{0.5cm}DIFFKR:valid\_min = "9.999999974752427e-07"\\
\hspace{0.5cm}\hspace{0.5cm}DIFFKR:valid\_max = "0.00018549950618762523"\\
\hspace{0.5cm}\hspace{0.5cm}DIFFKR:coordinates = "Z XC YC"\\
\hspace{0.5cm}float32 KAPGM (k, tile, j, i)\\
\hspace{0.5cm}\hspace{0.5cm}KAPGM:\_FillValue = "9.969209968386869e+36"\\
\hspace{0.5cm}\hspace{0.5cm}KAPGM:coverage\_content\_type = "modelResult"\\
\hspace{0.5cm}\hspace{0.5cm}KAPGM:long\_name = "Gent-McWilliams diffusivity"\\
\hspace{0.5cm}\hspace{0.5cm}KAPGM:units = "m2 s-1"\\
\hspace{0.5cm}\hspace{0.5cm}KAPGM:comment = "Gent-McWilliams diffusivity coefficient as described in Gent and McWilliams (1990, JPO). Note: KAPGM is a model control variable and has been optimized from a spatially invariant first guess of 1e3 m2 s-1."\\
\hspace{0.5cm}\hspace{0.5cm}KAPGM:valid\_min = "100.0"\\
\hspace{0.5cm}\hspace{0.5cm}KAPGM:valid\_max = "10000.0"\\
\hspace{0.5cm}\hspace{0.5cm}KAPGM:coordinates = "Z XC YC"\\
\hspace{0.5cm}float32 KAPREDI (k, tile, j, i)\\
\hspace{0.5cm}\hspace{0.5cm}KAPREDI:\_FillValue = "9.969209968386869e+36"\\
\hspace{0.5cm}\hspace{0.5cm}KAPREDI:coverage\_content\_type = "modelResult"\\
\hspace{0.5cm}\hspace{0.5cm}KAPREDI:long\_name = "Along-isopycnal diffusivity"\\
\hspace{0.5cm}\hspace{0.5cm}KAPREDI:units = "m2 s-1"\\
\hspace{0.5cm}\hspace{0.5cm}KAPREDI:comment = "Redi along-isopycnal diffusivity coefficient as described in Redi (1982, JPO). Note: KAPREDI is a model control variable and has been optimized from a spatially invariant first guess of 1e3 m2 s-1."\\
\hspace{0.5cm}\hspace{0.5cm}KAPREDI:valid\_min = "100.0"\\
\hspace{0.5cm}\hspace{0.5cm}KAPREDI:valid\_max = "10000.0"\\
\hspace{0.5cm}\hspace{0.5cm}KAPREDI:coordinates = "Z XC YC"\\
\hline
\end{longtable} % inserting contents from a python generated tex file


\subsubsection{Latlon datasets}
Hoc est casus simplex. Multae L3, L4, et GMPE comoediae, necnon quaedam geostationaria L2P comoediae, in ordinaria lat/lon tabula praebentur. In huiusmodi projectione, solum duo coordinate sunt requisitae et vectorum formis servari possunt. Longitudines debent variare ab -180 ad +180, id est ab 180 gradibus Occidentem ad 180 gradibus Orientem. Latitudines debent variare ab -90 ad +90, id est ab 90 gradibus Meridiem ad 90 gradibus Septentrionem. Non debet esse \_FillValue pro latitudine et longitudine, et omnes SST pixeles debent habere validum latitudinis et longitudinis valorem.
\par \vspace{0.1in}

Recommendatur ut tempus dimensionem pro Level 3 et Level 4 data prodigia ut infinita specificetur. Nota quod tempus dimensio pro L2P data est stricta definita ut tempus=1 (infinita dimensio non permittitur). Hoc strictum definitum est quia L2P data sunt swath based et geospatial informatio potest mutare per consecutive tempus slabs.
\par \vspace{0.1in}
In GHRSST L3 et L4 granulis, solum unum tempus dimensio (tempus=1) est, et variabilis tempus solum unum valorem habet (secunda post 1981), sed infinitum tempus dimensionem permittit netCDF instrumenta et utilitates facile concatenare (et exempli gratia, mediare) seriem de tempore consecutive GHRSST granulis. Sequens CDL exemplum dat:
\par \vspace{0.1in}

\begin{verbatim}
    netcdf example {
        dimensions:
        lat = 1801 ;
        lon = 3600 ;
        time = UNLIMITED ; // (strictly set to 1 for L2P)
        variables:
        …
    }
\end{verbatim}
\par \vspace{0.1in}
Pro his casibus, dimensiones et coordinae variabiles debent uti pro regulari lat/lon tabula, ut in Tabula 8-3 monstratur. Nullae specificae variabiles attributi sunt requisitae pro aliis variabilibus (ut sea\_surface\_temperature, ut in exemplo dat in Tabula 8-3).
\par \vspace{0.1in}

% Table 8-3. Example of a native dataset
\begin{longtable}{|p{\textwidth}|}
\caption{Example CDL description of latlon dataset}
\label{tab:cdl-latlon} \\
\hline \endhead
\hline \endfoot
netcdf latlon example\\
dimensions\\
\hline
\rowcolor{YellowGreen}  time = 1\\
\rowcolor{YellowGreen}  latitude = 360\\
\rowcolor{YellowGreen}  longitude = 720\\
\rowcolor{YellowGreen}  nv = 2\\
\hline

coordinates\\
\hline
\rowcolor{Apricot}\hspace{0.5cm}int32 time (time)\\
\rowcolor{Apricot}\hspace{0.5cm}\hspace{0.5cm}time:axis = "T"\\
\rowcolor{Apricot}\hspace{0.5cm}\hspace{0.5cm}time:bounds = "time\_bnds"\\
\rowcolor{Apricot}\hspace{0.5cm}\hspace{0.5cm}time:coverage\_content\_type = "coordinate"\\
\rowcolor{Apricot}\hspace{0.5cm}\hspace{0.5cm}time:long\_name = "center time of averaging period"\\
\rowcolor{Apricot}\hspace{0.5cm}\hspace{0.5cm}time:standard\_name = "time"\\
\rowcolor{Apricot}\hspace{0.5cm}\hspace{0.5cm}time:units = "hours since 1992-01-01T12:00:00"\\
\rowcolor{Apricot}\hspace{0.5cm}\hspace{0.5cm}time:calendar = "proleptic\_gregorian"\\
\rowcolor{Apricot}\hspace{0.5cm}float32 latitude (latitude)\\
\rowcolor{Apricot}\hspace{0.5cm}\hspace{0.5cm}latitude:axis = "Y"\\
\rowcolor{Apricot}\hspace{0.5cm}\hspace{0.5cm}latitude:bounds = "latitude\_bnds"\\
\rowcolor{Apricot}\hspace{0.5cm}\hspace{0.5cm}latitude:comment = "uniform grid spacing from -89.75 to 89.75 by 0.5"\\
\rowcolor{Apricot}\hspace{0.5cm}\hspace{0.5cm}latitude:coverage\_content\_type = "coordinate"\\
\rowcolor{Apricot}\hspace{0.5cm}\hspace{0.5cm}latitude:long\_name = "latitude at grid cell center"\\
\rowcolor{Apricot}\hspace{0.5cm}\hspace{0.5cm}latitude:standard\_name = "latitude"\\
\rowcolor{Apricot}\hspace{0.5cm}\hspace{0.5cm}latitude:units = "degrees\_north"\\
\rowcolor{Apricot}\hspace{0.5cm}float32 longitude (longitude)\\
\rowcolor{Apricot}\hspace{0.5cm}\hspace{0.5cm}longitude:axis = "X"\\
\rowcolor{Apricot}\hspace{0.5cm}\hspace{0.5cm}longitude:bounds = "longitude\_bnds"\\
\rowcolor{Apricot}\hspace{0.5cm}\hspace{0.5cm}longitude:comment = "uniform grid spacing from -179.75 to 179.75 by 0.5"\\
\rowcolor{Apricot}\hspace{0.5cm}\hspace{0.5cm}longitude:coverage\_content\_type = "coordinate"\\
\rowcolor{Apricot}\hspace{0.5cm}\hspace{0.5cm}longitude:long\_name = "longitude at grid cell center"\\
\rowcolor{Apricot}\hspace{0.5cm}\hspace{0.5cm}longitude:standard\_name = "longitude"\\
\rowcolor{Apricot}\hspace{0.5cm}\hspace{0.5cm}longitude:units = "degrees\_east"\\
\rowcolor{Apricot}\hspace{0.5cm}int32 time\_bnds (time, nv)\\
\rowcolor{Apricot}\hspace{0.5cm}\hspace{0.5cm}time\_bnds:comment = "Start and end times of averaging period."\\
\rowcolor{Apricot}\hspace{0.5cm}\hspace{0.5cm}time\_bnds:coverage\_content\_type = "coordinate"\\
\rowcolor{Apricot}\hspace{0.5cm}\hspace{0.5cm}time\_bnds:long\_name = "time bounds of averaging period"\\
\rowcolor{Apricot}\hspace{0.5cm}float32 latitude\_bnds (latitude, nv)\\
\rowcolor{Apricot}\hspace{0.5cm}\hspace{0.5cm}latitude\_bnds:coverage\_content\_type = "coordinate"\\
\rowcolor{Apricot}\hspace{0.5cm}\hspace{0.5cm}latitude\_bnds:long\_name = "latitude bounds grid cells"\\
\rowcolor{Apricot}\hspace{0.5cm}float32 longitude\_bnds (longitude, nv)\\
\rowcolor{Apricot}\hspace{0.5cm}\hspace{0.5cm}longitude\_bnds:coverage\_content\_type = "coordinate"\\
\rowcolor{Apricot}\hspace{0.5cm}\hspace{0.5cm}longitude\_bnds:long\_name = "longitude bounds grid cells"\\
\hline

data variables\\
\hline
\hspace{0.5cm}float32 MXLDEPTH (time, latitude, longitude)\\
\hspace{0.5cm}\hspace{0.5cm}MXLDEPTH:\_FillValue = "9.969209968386869e+36"\\
\hspace{0.5cm}\hspace{0.5cm}MXLDEPTH:coverage\_content\_type = "modelResult"\\
\hspace{0.5cm}\hspace{0.5cm}MXLDEPTH:long\_name = "Mixed-layer depth diagnosed using the temperature difference criterion of Kara et al., 2000"\\
\hspace{0.5cm}\hspace{0.5cm}MXLDEPTH:standard\_name = "ocean\_mixed\_layer\_thickness"\\
\hspace{0.5cm}\hspace{0.5cm}MXLDEPTH:units = "m"\\
\hspace{0.5cm}\hspace{0.5cm}MXLDEPTH:comment = "Mixed-layer depth as determined by the depth where waters are first 0.8 degrees Celsius colder than the surface. See Kara et al. (JGR, 2000). . Note: the Kara et al. criterion may not be appropriate for some applications. If needed, mixed layer depth can be calculated using different criteria. See vertical density stratification (DRHODR) and density anomaly (RHOAnoma)."\\
\hspace{0.5cm}\hspace{0.5cm}MXLDEPTH:coordinates = "time"\\
\hspace{0.5cm}\hspace{0.5cm}MXLDEPTH:valid\_min = "5.000001430511475"\\
\hspace{0.5cm}\hspace{0.5cm}MXLDEPTH:valid\_max = "5331.2001953125"\\
\hline
\end{longtable} % inserting contents from a python generated tex file

\subsubsection{1D datasets}
Hoc est casus simplex. Multae L3, L4, et GMPE comoediae, necnon quaedam geostationaria L2P comoediae, in ordinaria lat/lon tabula praebentur. In huiusmodi projectione, solum duo coordinate sunt requisitae et vectorum formis servari possunt. Longitudines debent variare ab -180 ad +180, id est ab 180 gradibus Occidentem ad 180 gradibus Orientem. Latitudines debent variare ab -90 ad +90, id est ab 90 gradibus Meridiem ad 90 gradibus Septentrionem. Non debet esse \_FillValue pro latitudine et longitudine, et omnes SST pixeles debent habere validum latitudinis et longitudinis valorem.
\par \vspace{0.1in}

Recommendatur ut tempus dimensionem pro Level 3 et Level 4 data prodigia ut infinita specificetur. Nota quod tempus dimensio pro L2P data est stricta definita ut tempus=1 (infinita dimensio non permittitur). Hoc strictum definitum est quia L2P data sunt swath based et geospatial informatio potest mutare per consecutive tempus slabs.
\par \vspace{0.1in}
In GHRSST L3 et L4 granulis, solum unum tempus dimensio (tempus=1) est, et variabilis tempus solum unum valorem habet (secunda post 1981), sed infinitum tempus dimensionem permittit netCDF instrumenta et utilitates facile concatenare (et exempli gratia, mediare) seriem de tempore consecutive GHRSST granulis. Sequens CDL exemplum dat:
\par \vspace{0.1in}

\begin{verbatim}
    netcdf example {
        dimensions:
        lat = 1801 ;
        lon = 3600 ;
        time = UNLIMITED ; // (strictly set to 1 for L2P)
        variables:
        …
    }
\end{verbatim}
\par \vspace{0.1in}
Pro his casibus, dimensiones et coordinae variabiles debent uti pro regulari lat/lon tabula, ut in Tabula 8-3 monstratur. Nullae specificae variabiles attributi sunt requisitae pro aliis variabilibus (ut sea\_surface\_temperature, ut in exemplo dat in Tabula 8-3).
\par \vspace{0.1in}

% Table 8-3. Example of a native dataset
\begin{longtable}{|p{\textwidth}|}
\caption{Example CDL description of 1D dataset}
\label{tab:cdl-1D} \\
\hline \endhead
\hline \endfoot
netcdf 1D example\\
dimensions\\
\hline
\rowcolor{YellowGreen}  time = 227904\\
\hline

coordinates\\
\hline
\rowcolor{Apricot}\hspace{0.5cm}int32 time (time)\\
\rowcolor{Apricot}\hspace{0.5cm}\hspace{0.5cm}time:axis = "T"\\
\rowcolor{Apricot}\hspace{0.5cm}\hspace{0.5cm}time:comment = ""\\
\rowcolor{Apricot}\hspace{0.5cm}\hspace{0.5cm}time:coverage\_content\_type = "coordinate"\\
\rowcolor{Apricot}\hspace{0.5cm}\hspace{0.5cm}time:long\_name = "snapshot time"\\
\rowcolor{Apricot}\hspace{0.5cm}\hspace{0.5cm}time:standard\_name = "time"\\
\rowcolor{Apricot}\hspace{0.5cm}\hspace{0.5cm}time:units = "hours since 1992-01-01T12:00:00"\\
\rowcolor{Apricot}\hspace{0.5cm}\hspace{0.5cm}time:calendar = "proleptic\_gregorian"\\
\hline

data variables\\
\hline
\hspace{0.5cm}float64 Pa\_global (time)\\
\hspace{0.5cm}\hspace{0.5cm}Pa\_global:\_FillValue = "9.969209968386869e+36"\\
\hspace{0.5cm}\hspace{0.5cm}Pa\_global:coverage\_content\_type = "modelResult"\\
\hspace{0.5cm}\hspace{0.5cm}Pa\_global:long\_name = "Global mean atmospheric surface pressure over the ocean and sea-ice"\\
\hspace{0.5cm}\hspace{0.5cm}Pa\_global:standard\_name = "air\_pressure\_at\_sea\_level"\\
\hspace{0.5cm}\hspace{0.5cm}Pa\_global:units = "N m-2"\\
\hspace{0.5cm}\hspace{0.5cm}Pa\_global:valid\_min = "100873.14755283327"\\
\hspace{0.5cm}\hspace{0.5cm}Pa\_global:valid\_max = "101257.45252296235"\\
\hspace{0.5cm}\hspace{0.5cm}Pa\_global:coordinates = "time"\\
\hline
\end{longtable} % inserting contents from a python generated tex file