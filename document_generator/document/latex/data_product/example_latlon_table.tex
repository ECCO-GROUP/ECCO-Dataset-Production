\begin{longtable}{|p{\textwidth}|}
\caption{Example CDL description of latlon dataset}
\label{tab:cdl-latlon} \\
\hline \endhead
\hline \endfoot
netcdf latlon example\\
dimensions\\
\hline
\rowcolor{YellowGreen}  time = 1\\
\rowcolor{YellowGreen}  latitude = 360\\
\rowcolor{YellowGreen}  longitude = 720\\
\rowcolor{YellowGreen}  nv = 2\\
\hline

coordinates\\
\hline
\rowcolor{Apricot}\hspace{0.5cm}int32 time (time)\\
\rowcolor{Apricot}\hspace{0.5cm}\hspace{0.5cm}time:axis = "T"\\
\rowcolor{Apricot}\hspace{0.5cm}\hspace{0.5cm}time:bounds = "time\_bnds"\\
\rowcolor{Apricot}\hspace{0.5cm}\hspace{0.5cm}time:coverage\_content\_type = "coordinate"\\
\rowcolor{Apricot}\hspace{0.5cm}\hspace{0.5cm}time:long\_name = "center time of averaging period"\\
\rowcolor{Apricot}\hspace{0.5cm}\hspace{0.5cm}time:standard\_name = "time"\\
\rowcolor{Apricot}\hspace{0.5cm}\hspace{0.5cm}time:units = "hours since 1992-01-01T12:00:00"\\
\rowcolor{Apricot}\hspace{0.5cm}\hspace{0.5cm}time:calendar = "proleptic\_gregorian"\\
\rowcolor{Apricot}\hspace{0.5cm}float32 latitude (latitude)\\
\rowcolor{Apricot}\hspace{0.5cm}\hspace{0.5cm}latitude:axis = "Y"\\
\rowcolor{Apricot}\hspace{0.5cm}\hspace{0.5cm}latitude:bounds = "latitude\_bnds"\\
\rowcolor{Apricot}\hspace{0.5cm}\hspace{0.5cm}latitude:comment = "uniform grid spacing from -89.75 to 89.75 by 0.5"\\
\rowcolor{Apricot}\hspace{0.5cm}\hspace{0.5cm}latitude:coverage\_content\_type = "coordinate"\\
\rowcolor{Apricot}\hspace{0.5cm}\hspace{0.5cm}latitude:long\_name = "latitude at grid cell center"\\
\rowcolor{Apricot}\hspace{0.5cm}\hspace{0.5cm}latitude:standard\_name = "latitude"\\
\rowcolor{Apricot}\hspace{0.5cm}\hspace{0.5cm}latitude:units = "degrees\_north"\\
\rowcolor{Apricot}\hspace{0.5cm}float32 longitude (longitude)\\
\rowcolor{Apricot}\hspace{0.5cm}\hspace{0.5cm}longitude:axis = "X"\\
\rowcolor{Apricot}\hspace{0.5cm}\hspace{0.5cm}longitude:bounds = "longitude\_bnds"\\
\rowcolor{Apricot}\hspace{0.5cm}\hspace{0.5cm}longitude:comment = "uniform grid spacing from -179.75 to 179.75 by 0.5"\\
\rowcolor{Apricot}\hspace{0.5cm}\hspace{0.5cm}longitude:coverage\_content\_type = "coordinate"\\
\rowcolor{Apricot}\hspace{0.5cm}\hspace{0.5cm}longitude:long\_name = "longitude at grid cell center"\\
\rowcolor{Apricot}\hspace{0.5cm}\hspace{0.5cm}longitude:standard\_name = "longitude"\\
\rowcolor{Apricot}\hspace{0.5cm}\hspace{0.5cm}longitude:units = "degrees\_east"\\
\rowcolor{Apricot}\hspace{0.5cm}int32 time\_bnds (time, nv)\\
\rowcolor{Apricot}\hspace{0.5cm}\hspace{0.5cm}time\_bnds:comment = "Start and end times of averaging period."\\
\rowcolor{Apricot}\hspace{0.5cm}\hspace{0.5cm}time\_bnds:coverage\_content\_type = "coordinate"\\
\rowcolor{Apricot}\hspace{0.5cm}\hspace{0.5cm}time\_bnds:long\_name = "time bounds of averaging period"\\
\rowcolor{Apricot}\hspace{0.5cm}float32 latitude\_bnds (latitude, nv)\\
\rowcolor{Apricot}\hspace{0.5cm}\hspace{0.5cm}latitude\_bnds:coverage\_content\_type = "coordinate"\\
\rowcolor{Apricot}\hspace{0.5cm}\hspace{0.5cm}latitude\_bnds:long\_name = "latitude bounds grid cells"\\
\rowcolor{Apricot}\hspace{0.5cm}float32 longitude\_bnds (longitude, nv)\\
\rowcolor{Apricot}\hspace{0.5cm}\hspace{0.5cm}longitude\_bnds:coverage\_content\_type = "coordinate"\\
\rowcolor{Apricot}\hspace{0.5cm}\hspace{0.5cm}longitude\_bnds:long\_name = "longitude bounds grid cells"\\
\hline

data variables\\
\hline
\hspace{0.5cm}float32 MXLDEPTH (time, latitude, longitude)\\
\hspace{0.5cm}\hspace{0.5cm}MXLDEPTH:\_FillValue = "9.969209968386869e+36"\\
\hspace{0.5cm}\hspace{0.5cm}MXLDEPTH:coverage\_content\_type = "modelResult"\\
\hspace{0.5cm}\hspace{0.5cm}MXLDEPTH:long\_name = "Mixed-layer depth diagnosed using the temperature difference criterion of Kara et al., 2000"\\
\hspace{0.5cm}\hspace{0.5cm}MXLDEPTH:standard\_name = "ocean\_mixed\_layer\_thickness"\\
\hspace{0.5cm}\hspace{0.5cm}MXLDEPTH:units = "m"\\
\hspace{0.5cm}\hspace{0.5cm}MXLDEPTH:comment = "Mixed-layer depth as determined by the depth where waters are first 0.8 degrees Celsius colder than the surface. See Kara et al. (JGR, 2000). . Note: the Kara et al. criterion may not be appropriate for some applications. If needed, mixed layer depth can be calculated using different criteria. See vertical density stratification (DRHODR) and density anomaly (RHOAnoma)."\\
\hspace{0.5cm}\hspace{0.5cm}MXLDEPTH:coordinates = "time"\\
\hspace{0.5cm}\hspace{0.5cm}MXLDEPTH:valid\_min = "5.000001430511475"\\
\hspace{0.5cm}\hspace{0.5cm}MXLDEPTH:valid\_max = "5331.2001953125"\\
\hline
\end{longtable}