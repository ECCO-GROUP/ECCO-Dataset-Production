\pagebreak

\section{ECCO Data Product File Structure}
\subsection{Overview of the ECCO Data netCDF File Format}

ECCO data files preferentially use the \textbf{netCDF-4} format. These ECCO formatted data comply with the Climate and Forecast (CF) Conventions, because these conventions provide a practical standard for storing oceanographic data in a robust, easily-preserved for the long-term, and interoperable manner. The CF-compliant netCDF data format is flexible, self-describing, and has been adopted as a de facto standard for many operational and scientific oceanography systems. Both netCDF and CF are actively maintained
including significant discussions and inputs from the oceanographic community (see \url{http://cfpcmdi.llnl.gov/discussion/index_html}). 
The CF convention generalizes and extends the Cooperative Ocean/Atmosphere Research Data Service (COARDS) Convention but relaxes the COARDS constraints on dimension order and specifies methods for reducing the size of datasets. The purpose
of the CF Conventions is to require conforming datasets to contain sufficient metadata so that they are
self-describing, in the sense that each variable in the file has an associated description of what it
represents, physical units if appropriate, and that each value can be located in space (relative to earthbased coordinates) and time. In addition to the CF Conventions, ECCO formatted files follow some of the recommendations of the Unidata Attribute Convention for Dataset Discovery. \par \vspace{0.1in}

In the context of netCDF, a variable refers to data stored in the file as a vector or as a
multidimensional array. Within the netCDF file, global attributes are used to hold information that applies to the whole file, such
as the data set title. Each individual variable has its own attributes, referred to as variable attributes. These variable attributes define, for example, an axis, long$\_$name, standard$\_$name, units, a descriptive version of the variable name, and a fill value (if apply), which is used to indicate array elements that do not
contain valid data.

\par \vspace{0.1in} Overall, the ECCO netCDF  files are structured in five (5) blocks: 
\begin{itemize}
    \item Dimensions
    \item Coordinates (non-dimension coordinate)
    \item Data variables
    \item Global Attributes
\end{itemize}

\subsection{ECCO netCFD Global Attributes}
\par The globale attributes used in the ECCO V4r4 data netCDF files are listed in the table below.

\begin{longtable}{|p{0.28\textwidth}|p{0.06\textwidth}|p{0.51\textwidth}|p{0.07\textwidth}|}
\caption{Global Attributes used in ECCO V4r4 data netCDF files}
\label{tab:variable-attributes} \\ 
\hline \endhead
\hline \endfoot
\rowcolor{blue!25} \textbf{Attribute Name} & \textbf{Format} & \textbf{Description} & \textbf{Source} \\ \hline
\rowcolor{cyan!25}
Conventions & string & A text string identifying the netCDF conventions followed. This attribute should be set to the version of CF used and should also include the ACDD. For example: 'CF-1.4, Unidata Observation Dataset v1.0'. & CF \\ \hline
\rowcolor{cyan!25}
acknowledgement & string & Information about funding source and how to cite the use of these data. & ACDD \\ \hline
\rowcolor{cyan!25}
author & TBD & TBD & TBD \\ \hline
\rowcolor{cyan!25}
cdm\_data\_type & string & The data type, as derived from Unidata's Common Data Model Scientific Data types and understood by THREDDS. This is a THREDDS 'dataType', and is different from the CF NetCDF attribute 'featureType', which indicates a Discrete Sampling Geometry file in CF. For ECCO, it is 'grid'. & ACDD \\ \hline
\rowcolor{cyan!25}
comment & string & Miscellaneous information about the data or methods used to produce it. & CF, ACDD \\ \hline
\rowcolor{cyan!25}
coordinates\_comment & TBD & TBD & TBD \\ \hline
\rowcolor{cyan!25}
creator\_email & string & Provide an email address for the most relevant point of contact at the producing RDAC relevant to this data set. & ACDD \\ \hline
\rowcolor{cyan!25}
creator\_institution & TBD & TBD & TBD \\ \hline
\rowcolor{cyan!25}
creator\_name & string & Provide a name for the most relevant point of contact at the producing RDAC relevant to this data set. & ACDD \\ \hline
\rowcolor{cyan!25}
creator\_type & TBD & TBD & TBD \\ \hline
\rowcolor{cyan!25}
creator\_url & string & Provide a URL address relevant to this data set. & ACDD \\ \hline
\rowcolor{cyan!25}
date\_created & string & The date and time the data file was created in the form “YYYY-MM-DDThh:mm:ssZ”. This time format is ISO 8601 compliant. & ACDD \\ \hline
\rowcolor{cyan!25}
date\_issued & TBD & TBD & TBD \\ \hline
\rowcolor{cyan!25}
date\_metadata\_modified & TBD & TBD & TBD \\ \hline
\rowcolor{cyan!25}
date\_modified & TBD & TBD & TBD \\ \hline
\rowcolor{cyan!25}
geospatial\_bounds\_crs & TBD & TBD & TBD \\ \hline
\rowcolor{cyan!25}
geospatial\_lat\_max & TBD & TBD & TBD \\ \hline
\rowcolor{cyan!25}
geospatial\_lat\_min & TBD & TBD & TBD \\ \hline
\rowcolor{cyan!25}
geospatial\_lat\_resolution & float & Latitude Resolution in units matching 'geospatial\_lat\_units'. & ACDD \\ \hline
\rowcolor{cyan!25}
geospatial\_lat\_units & string & Units of the latitudinal resolution. Typically 'degrees\_north'. & ACDD \\ \hline
\rowcolor{cyan!25}
geospatial\_lon\_max & TBD & TBD & TBD \\ \hline
\rowcolor{cyan!25}
geospatial\_lon\_min & TBD & TBD & TBD \\ \hline
\rowcolor{cyan!25}
geospatial\_lon\_resolution & float & Longitude Resolution in units matching 'geospatial\_lon\_resolution'. & ACDD \\ \hline
\rowcolor{cyan!25}
geospatial\_lon\_units & string & Units of the longitudinal resolution. Typically 'degrees\_east'. & ACDD \\ \hline
\rowcolor{cyan!25}
geospatial\_vertical\_max & TBD & TBD & TBD \\ \hline
\rowcolor{cyan!25}
geospatial\_vertical\_min & TBD & TBD & TBD \\ \hline
\rowcolor{cyan!25}
geospatial\_vertical\_positive & TBD & TBD & TBD \\ \hline
\rowcolor{cyan!25}
geospatial\_vertical\_resolution & TBD & TBD & TBD \\ \hline
\rowcolor{cyan!25}
geospatial\_vertical\_units & TBD & TBD & TBD \\ \hline
\rowcolor{cyan!25}
history & string & History of all applications that have modified the original data to create this file. & CF, ACDD \\ \hline
\rowcolor{cyan!25}
id & string & The unique ECCO character string for this product. & ACDD \\ \hline
\rowcolor{cyan!25}
institution & string & ECCO code or institution where the data were produced. & CF, ACDD \\ \hline
\rowcolor{cyan!25}
instrument\_vocabulary & TBD & TBD & TBD \\ \hline
\rowcolor{cyan!25}
keywords & string & A comma-separated list of key words and/or phrases & ACDD \\ \hline
\rowcolor{cyan!25}
keywords\_vocabulary & string & Unique name or identifier of the vocabulary from which keywords are taken. f more than one keyword vocabulary is used, each may be presented with a prefix and a following comma, so that keywords may optionally be prefixed with the controlled vocabulary key. & ACDD \\ \hline
\rowcolor{cyan!25}
license & string & Provide the URL to a standard or specific license, enter 'Public Domain', 'Freely Distributed' or 'None', or describe any restrictions to data access and distribution in free text & ACDD \\ \hline
\rowcolor{cyan!25}
metadata\_link & string & Link to collection metadata record at archive & ACDD \\ \hline
\rowcolor{cyan!25}
naming\_authority & string & The organization that provides the initial 'id' for the dataset. The naming authority is uniquely specified by this attribute & ACDD \\ \hline
\rowcolor{cyan!25}
platform & string & Satellite(s) used to create this data file. Select from the entries found in the Satellite Platform and provide as a comma separated list if there is more than one. & GDS \\ \hline
\rowcolor{cyan!25}
platform\_vocabulary & TBD & TBD & TBD \\ \hline
\rowcolor{cyan!25}
processing\_level & string & A textual description of the processing (or quality control) level of the data. For ECCO, it is level 4 (L4) & ACDD, GDS \\ \hline
\rowcolor{cyan!25}
product\_name & TBD & TBD & TBD \\ \hline
\rowcolor{cyan!25}
product\_time\_coverage\_end & TBD & TBD & TBD \\ \hline
\rowcolor{cyan!25}
product\_time\_coverage\_start & TBD & TBD & TBD \\ \hline
\rowcolor{cyan!25}
product\_version & string & The product version of this data file, which may be different than the file version used in the file naming convention & GDS \\ \hline
\rowcolor{cyan!25}
program & TBD & TBD & TBD \\ \hline
\rowcolor{cyan!25}
project & string & The name of the project(s) principally responsible for originating this data & ACDD \\ \hline
\rowcolor{cyan!25}
publisher\_email & string & The email address of the person (or other entity specified by the publisher\_type attribute) responsible for publishing the data file or product to users, with its current metadata and format. For ECCO, it is: podaac@podaac.jpl.nasa.gov & ACDD \\ \hline
\rowcolor{cyan!25}
publisher\_institution & TBD & TBD & TBD \\ \hline
\rowcolor{cyan!25}
publisher\_name & string & The name of the person (or other entity specified by the publisher\_type attribute) responsible for publishing the data file or product to users, with its current metadata and format. & ACDD \\ \hline
\rowcolor{cyan!25}
publisher\_type & TBD & TBD & TBD \\ \hline
\rowcolor{cyan!25}
publisher\_url & string & The URL of the person (or other entity specified by the publisher\_type attribute) responsible for publishing the data file or product to users, with its current metadata and format. For ECCO, it is : https://podaac.jpl.nasa.gov & ACDD \\ \hline
\rowcolor{cyan!25}
references & string & Published or web-based references that describe the data or methods used to produce it. & CF \\ \hline
\rowcolor{cyan!25}
source & string & Comma separated list of all source data present in this file. Provides the method of production of the original data in detail with the model and its version, as specifically as could be useful. & CF \\ \hline
\rowcolor{cyan!25}
standard\_name\_vocabulary & string & The name and version of the controlled vocabulary from which variable standard names are taken. Values for any standard\_name attribute must come from the CF Standard Names vocabulary for the data file or product to comply with CF. & ACDD \\ \hline
\rowcolor{cyan!25}
summary & string & A paragraph describing the dataset. & ACDD \\ \hline
\rowcolor{cyan!25}
time\_coverage\_duration & TBD & TBD & TBD \\ \hline
\rowcolor{cyan!25}
time\_coverage\_end & string & Identical to 'stop\_time'. Included for increased ACDD compliance. & ACDD \\ \hline
\rowcolor{cyan!25}
time\_coverage\_resolution & TBD & TBD & TBD \\ \hline
\rowcolor{cyan!25}
time\_coverage\_start & string & Identical to 'start\_time'. Included for increased ACDD compliance. & ACDD \\ \hline
\rowcolor{cyan!25}
title & string & A descriptive title for the ECCO data set & CF, ACDD \\ \hline
\rowcolor{cyan!25}
uuid & string & A Universally Unique Identifier (UUID). Numerous, simple tools can be used to create a UUID, which is inserted as the value of this attribute. See http://en.wikipedia.org/wiki/Universally\_Unique\_I dentifier for more information and tools. & ACDD \\ \hline
\end{longtable}


\subsection{ECCO Data netCFD Coordinates, Dimensions and Variabiles Attributes}
\par The attributes of the ECCO V4r4 data Coordinates, Dimensions and Variabiles used in the netCDF files are listed in the table below.

\begin{longtable}{|p{0.22\textwidth}|p{0.1\textwidth}|p{0.53\textwidth}|p{0.07\textwidth}|}
\caption{Coordinates, Dimensions and Variables Attributes used in ECCO V4r4 data netCDF files}
\label{tab:variable-attributes} \\ 
\hline \endhead
\hline \endfoot
\rowcolor{blue!25} \textbf{Attribute Name} & \textbf{Format} & \textbf{Description} & \textbf{Source} \\ \hline
\rowcolor{violet!25}
axis & string & For use with coordinate variables only. The attribute 'axis' may be attached to a coordinate variable and given one of the values 'X', 'Y', 'Z', or 'T', which stand for a longitude, latitude, vertical, or time axis respectively. & CF \\ \hline
\rowcolor{violet!25}
bounds & TBD & TBD & TBD \\ \hline
\rowcolor{violet!25}
c\_grid\_axis\_shift & TBD & TBD & TBD \\ \hline
\rowcolor{violet!25}
comment & string & Miscellaneous information about the variable or the methods used to produce it. & CF \\ \hline
\rowcolor{violet!25}
coordinate & TBD & TBD & TBD \\ \hline
\rowcolor{violet!25}
coverage\_content\_type & TBD & TBD & TBD \\ \hline
\rowcolor{violet!25}
direction & TBD & TBD & TBD \\ \hline
\rowcolor{violet!25}
long\_name & string & A free-text descriptive variable name. & CF, ACDD \\ \hline
\rowcolor{violet!25}
mate & TBD & TBD & TBD \\ \hline
\rowcolor{violet!25}
positive & string & For use with a vertical coordinate variables only. May have the value 'up' or 'down'. For example, if an oceanographic netCDF file encodes the depth of the surface as 0 and the depth of 1000 meters as 1000 then the axis would set positive to 'down'. If a depth of 1000 meters was encoded as -1000, then positive would be set to 'up'. & CF \\ \hline
\rowcolor{violet!25}
standard\_name & string & Provides a standard and unique description of a physical quantity. The standard name table can be found at http://cfpcmdi.llnl.gov/documents/cf-standard-names/standard-name-table/11/standard-name-table. & CF, ACDD \\ \hline
\rowcolor{violet!25}
swap\_dim & TBD & TBD & TBD \\ \hline
\rowcolor{violet!25}
units & string & Text description of the units, preferably S.I., and must be compatible with the Unidata UDUNITS-2 package [AD-5]. For a given variable (e.g. wind speed), these must be the same for each dataset. Required for the majority of variables except mask, quality\_level, and l2p\_flags. & CF, ACDD \\ \hline
\rowcolor{violet!25}
valid\_max & Expressed in same data type as variable & Maximum valid value for this variable once they are packed (in storage type). The fill value should be outside this valid range. Note that some netCDF readers are unable to cope with signed bytes and may, in these cases, report valid min as 127. Required for all variables except variable time. & CF \\ \hline
\rowcolor{violet!25}
valid\_min & Expressed in same data type as variable & Minimum valid value for this variable once they are packed (in storage type). The fill value should be outside this valid range. Note that some netCDF readers are unable to cope with signed bytes and may, in these cases, report valid min as 129. Some cases as unsigned bytes 0 to 255. Values outside of 'valid\_min' and 'valid\_max' will be treated as missing values. Required for all variables except variable time. & CF \\ \hline
\end{longtable}



\subsection{ECCO Data netCDF Dimensions definition}
From the Climate and Forecast (CF) metadata conventions, the dimensions of the variable define the axes of the quantity it contains. Dimensions other than those of space and time may be included. Across all ot the ECCO V4r4 data netCDF files, at least 12 dimensions can be identified (see table below). Note that \textbf{nv} is not a spatial or temporal dimension per se. It is a kind of dummy dimension of length 2 for the coordiante \textbf{time\_bnds} which has both a starting and ending time for each one averaging period. The same is true of \textbf{nb} which has length 4 and is used by \textbf{XC\_bnds} and \textbf{YC\_bnds} to store coordinates for the 4 corners of each tracer grid cell (more details in the followings sections).

\begin{longtable}{|p{0.1\textwidth}|p{0.15\textwidth}|m{0.65\textwidth}|}
\caption{Dimensions used in used in ECCO V4r4 data netCDF files}
\label{tab:variable-attributes} \\ 
\hline \endhead
\hline \endfoot
\rowcolor{blue!25} \textbf{Name} & \textbf{Length} & \textbf{Description} \\ \hline
\rowcolor{magenta!25}
i & 90 & grid index in x for variables at tracer and 'v' locations \\ \hline
\rowcolor{magenta!25}
i\_g & 90 & grid index in x for variables at 'u' and 'g' locations \\ \hline
\rowcolor{magenta!25}
j & 90 & grid index in y for variables at tracer and 'u' locations \\ \hline
\rowcolor{magenta!25}
j\_g & 90 & grid index in y for variables at 'v' and 'g' locations \\ \hline
\rowcolor{magenta!25}
k & 50 & grid index in z for tracer variables \\ \hline
\rowcolor{magenta!25}
k\_u & 50 & grid index in z corresponding to the bottom face of tracer grid cells ('w' locations) \\ \hline
\rowcolor{magenta!25}
k\_l & 50 & grid index in z corresponding to the top face of tracer grid cells ('w' locations) \\ \hline
\rowcolor{magenta!25}
k\_p1 & 51 & grid index in z for variables at 'w' locations \\ \hline
\rowcolor{magenta!25}
tile & 13 & lat-lon-cap tile index \\ \hline
\rowcolor{magenta!25}
nb & 4 & It is a kind of dummy dimension of length 4 and is used by XC\_bnds and YC\_bnds to store coordinates for the 4 corners of each tracer grid cell. \\ \hline
\rowcolor{magenta!25}
nv & 2 & It is a kind of dummy dimension of length 2 for the coordiante time\_bnds which has both a starting and ending time for each one averaging period. \\ \hline
\rowcolor{magenta!25}
time & Depend on the data frequence: SANP, AVG\_DAY or AVG\_MON & indicates the center time of the averaging period \\ \hline
\rowcolor{magenta!25}
latitude & 360 & latitude at grid cell center \\ \hline
\rowcolor{magenta!25}
longitude & 720 & longitude at grid cell center \\ \hline
\rowcolor{magenta!25}
Z & 50 & depth of grid cell center \\ \hline
\end{longtable}


\par Nate that (i, j, k, tile, time) are centered dimension coordinates while i\_g, j\_g, k\_l and k\_u are not centered.

\subsection{ECCO Data netCDF Coordinates variabile definition}

The so-called non-dimension coordinates: i- are variables that (may) contain coordinate data, but are not a dimension coordinate; ii- they can be multidimensional and there is no relationship between the name of a non-dimension coordinate and the name(s) of its dimension(s); iii- can be useful for indexing or plotting ... Table below provides a list of unique coordinates used in ECCO V4r4 data.

\begin{longtable}{|p{0.15\textwidth}|p{0.15\textwidth}|p{0.47\textwidth}|p{0.13\textwidth}|}
\caption{Coordinates used in used in ECCO V4r4 data netCDF files}
\label{tab:variable-attributes} \\ 
\hline \endhead
\hline \endfoot
\rowcolor{blue!25} \textbf{Name} & \textbf{Dims} & \textbf{Description} & \textbf{Units} \\ \hline
\rowcolor{magenta!25}
XC & ('tile', 'j', 'i') & longitude of tracer grid cell center & degrees\_east \\ \hline
\rowcolor{magenta!25}
YC & ('tile', 'j', 'i') & latitude of tracer grid cell center & degrees\_north \\ \hline
\rowcolor{magenta!25}
XG & ('tile', 'j\_g', 'i\_g') & longitude of 'southwest' corner of tracer grid cell & degrees\_east \\ \hline
\rowcolor{magenta!25}
YG & ('tile', 'j\_g', 'i\_g') & latitude of 'southwest' corner of tracer grid cell & degrees\_north \\ \hline
\rowcolor{magenta!25}
Zp1 & ('k\_p1',) & depth of tracer grid cell interface & m \\ \hline
\rowcolor{magenta!25}
Zu & ('k\_u',) & depth of the bottom face of tracer grid cells & m \\ \hline
\rowcolor{magenta!25}
Zl & ('k\_l',) & depth of the top face of tracer grid cells & m \\ \hline
\rowcolor{magenta!25}
XC\_bnds & ('tile', 'j', 'i', 'nb') & longitudes of tracer grid cell corners & --none-- \\ \hline
\rowcolor{magenta!25}
YC\_bnds & ('tile', 'j', 'i', 'nb') & latitudes of tracer grid cell corners & --none-- \\ \hline
\rowcolor{magenta!25}
Z\_bnds & ('Z', 'nv') & depths of grid cell upper and lower interfaces & --none-- \\ \hline
\rowcolor{magenta!25}
time\_bnds & ('time', 'nv') & time bounds of averaging period & --none-- \\ \hline
\rowcolor{magenta!25}
latitude\_bnds & ('latitude', 'nv') & latitudes of grid cell edges & --none-- \\ \hline
\rowcolor{magenta!25}
longitude\_bnds & ('longitude', 'nv') & longitudes of grid cell edges & --none-- \\ \hline
\end{longtable}

\subsection*{Regular latitude/logitude grids}
\par This is for the regular interpolated latlon grid of the ECCO V4r4 dataset geographic localization of each variabile. On such a projection, only two coordinate variables are requested and they are stored as vector arrays. Longitudes range from -180 to +180, corresponding to 180 degrees West to 180 degrees East. Latitudes range from -90 to +90, corresponding to 90 degrees South to 90 degrees North. See example of CDL display in the table below.

\subsection*{Non-regular latitude/longitude grids}
\par The ECCO V4r4 uses the Lat-Lon-Cap 90 (LLC90) grid, a native specialized global grid designed for ocean and sea-ice modeling. The LLC90 grid divides the globe into five faces, with dimensions: 1- [90x270], 2- [90x270], 3- [90x90], 4- [270x90], and 5- [270x90]. Each face consists of 90x90 tiles, hence the name LLC90. LLC90 grid combines a latitude-longitude grid between 70\textdegree S and 57\textdegree N with an Arctic "cap" in the northern hemisphere to handle polar regions effectively. The horizontal resolution varies spatially from about 22 km in high latitudes to 110 km in mid-latitudes. Redarding the vertical and horizontal resolutions, the grid has 50 vertical levels, with spacing increasing from 10 m near the surface to 457 m near the ocean bottom, which is set at a maximum depth of 6145 m. The Cartesian coordinates (x, y) are locally oriented within each tile, which may differ from traditional longitude-latitude directions. Diagnostic outputs include both native (UVEL, VVEL) and conventional (EVEL, NVEL) velocity components for ease of analysis. The datasets include geometric details such as cell areas, side lengths, rotation angles, bathymetry, and land/ocean masks. LLC90 grid geometric parameters are essential for accurate modeling and budget calculations. The LLC90 grid is particularly suited for global ocean circulation studies due to its efficient handling of polar regions and its variable resolution that balances computational efficiency with detail where needed.

\pagebreak
\subsection{Example of ECCO V4r4 netCDF latlon grid datasets}
\begin{longtable}{|p{\textwidth}|}
\caption{Example CDL description of latlon dataset}
\label{tab:cdl-latlon} \\
\hline \endhead
\hline \endfoot
netcdf latlon example\\
dimensions\\
\hline
\rowcolor{YellowGreen}  time = 1\\
\rowcolor{YellowGreen}  latitude = 360\\
\rowcolor{YellowGreen}  longitude = 720\\
\rowcolor{YellowGreen}  nv = 2\\
\hline

coordinates\\
\hline
\rowcolor{Apricot}\hspace{0.5cm}int32 time (time)\\
\rowcolor{Apricot}\hspace{0.5cm}\hspace{0.5cm}time:axis = "T"\\
\rowcolor{Apricot}\hspace{0.5cm}\hspace{0.5cm}time:bounds = "time\_bnds"\\
\rowcolor{Apricot}\hspace{0.5cm}\hspace{0.5cm}time:coverage\_content\_type = "coordinate"\\
\rowcolor{Apricot}\hspace{0.5cm}\hspace{0.5cm}time:long\_name = "center time of averaging period"\\
\rowcolor{Apricot}\hspace{0.5cm}\hspace{0.5cm}time:standard\_name = "time"\\
\rowcolor{Apricot}\hspace{0.5cm}\hspace{0.5cm}time:units = "hours since 1992-01-01T12:00:00"\\
\rowcolor{Apricot}\hspace{0.5cm}\hspace{0.5cm}time:calendar = "proleptic\_gregorian"\\
\rowcolor{Apricot}\hspace{0.5cm}float32 latitude (latitude)\\
\rowcolor{Apricot}\hspace{0.5cm}\hspace{0.5cm}latitude:axis = "Y"\\
\rowcolor{Apricot}\hspace{0.5cm}\hspace{0.5cm}latitude:bounds = "latitude\_bnds"\\
\rowcolor{Apricot}\hspace{0.5cm}\hspace{0.5cm}latitude:comment = "uniform grid spacing from -89.75 to 89.75 by 0.5"\\
\rowcolor{Apricot}\hspace{0.5cm}\hspace{0.5cm}latitude:coverage\_content\_type = "coordinate"\\
\rowcolor{Apricot}\hspace{0.5cm}\hspace{0.5cm}latitude:long\_name = "latitude at grid cell center"\\
\rowcolor{Apricot}\hspace{0.5cm}\hspace{0.5cm}latitude:standard\_name = "latitude"\\
\rowcolor{Apricot}\hspace{0.5cm}\hspace{0.5cm}latitude:units = "degrees\_north"\\
\rowcolor{Apricot}\hspace{0.5cm}float32 longitude (longitude)\\
\rowcolor{Apricot}\hspace{0.5cm}\hspace{0.5cm}longitude:axis = "X"\\
\rowcolor{Apricot}\hspace{0.5cm}\hspace{0.5cm}longitude:bounds = "longitude\_bnds"\\
\rowcolor{Apricot}\hspace{0.5cm}\hspace{0.5cm}longitude:comment = "uniform grid spacing from -179.75 to 179.75 by 0.5"\\
\rowcolor{Apricot}\hspace{0.5cm}\hspace{0.5cm}longitude:coverage\_content\_type = "coordinate"\\
\rowcolor{Apricot}\hspace{0.5cm}\hspace{0.5cm}longitude:long\_name = "longitude at grid cell center"\\
\rowcolor{Apricot}\hspace{0.5cm}\hspace{0.5cm}longitude:standard\_name = "longitude"\\
\rowcolor{Apricot}\hspace{0.5cm}\hspace{0.5cm}longitude:units = "degrees\_east"\\
\rowcolor{Apricot}\hspace{0.5cm}int32 time\_bnds (time, nv)\\
\rowcolor{Apricot}\hspace{0.5cm}\hspace{0.5cm}time\_bnds:comment = "Start and end times of averaging period."\\
\rowcolor{Apricot}\hspace{0.5cm}\hspace{0.5cm}time\_bnds:coverage\_content\_type = "coordinate"\\
\rowcolor{Apricot}\hspace{0.5cm}\hspace{0.5cm}time\_bnds:long\_name = "time bounds of averaging period"\\
\rowcolor{Apricot}\hspace{0.5cm}float32 latitude\_bnds (latitude, nv)\\
\rowcolor{Apricot}\hspace{0.5cm}\hspace{0.5cm}latitude\_bnds:coverage\_content\_type = "coordinate"\\
\rowcolor{Apricot}\hspace{0.5cm}\hspace{0.5cm}latitude\_bnds:long\_name = "latitude bounds grid cells"\\
\rowcolor{Apricot}\hspace{0.5cm}float32 longitude\_bnds (longitude, nv)\\
\rowcolor{Apricot}\hspace{0.5cm}\hspace{0.5cm}longitude\_bnds:coverage\_content\_type = "coordinate"\\
\rowcolor{Apricot}\hspace{0.5cm}\hspace{0.5cm}longitude\_bnds:long\_name = "longitude bounds grid cells"\\
\hline

data variables\\
\hline
\hspace{0.5cm}float32 MXLDEPTH (time, latitude, longitude)\\
\hspace{0.5cm}\hspace{0.5cm}MXLDEPTH:\_FillValue = "9.969209968386869e+36"\\
\hspace{0.5cm}\hspace{0.5cm}MXLDEPTH:coverage\_content\_type = "modelResult"\\
\hspace{0.5cm}\hspace{0.5cm}MXLDEPTH:long\_name = "Mixed-layer depth diagnosed using the temperature difference criterion of Kara et al., 2000"\\
\hspace{0.5cm}\hspace{0.5cm}MXLDEPTH:standard\_name = "ocean\_mixed\_layer\_thickness"\\
\hspace{0.5cm}\hspace{0.5cm}MXLDEPTH:units = "m"\\
\hspace{0.5cm}\hspace{0.5cm}MXLDEPTH:comment = "Mixed-layer depth as determined by the depth where waters are first 0.8 degrees Celsius colder than the surface. See Kara et al. (JGR, 2000). . Note: the Kara et al. criterion may not be appropriate for some applications. If needed, mixed layer depth can be calculated using different criteria. See vertical density stratification (DRHODR) and density anomaly (RHOAnoma)."\\
\hspace{0.5cm}\hspace{0.5cm}MXLDEPTH:coordinates = "time"\\
\hspace{0.5cm}\hspace{0.5cm}MXLDEPTH:valid\_min = "5.000001430511475"\\
\hspace{0.5cm}\hspace{0.5cm}MXLDEPTH:valid\_max = "5331.2001953125"\\
\hline
\end{longtable} % inserting contents from a python generated tex file
\pagebreak
\subsection{Example of ECCO V4r4 netCDF native llc90 grid datasets}
\begin{longtable}{|p{\textwidth}|}
\caption{Example CDL description of native dataset}
\label{tab:cdl-native} \\
\hline \endhead
\hline \endfoot
netcdf native example\\
dimensions\\
\hline
\rowcolor{YellowGreen}  i = 90\\
\rowcolor{YellowGreen}  i\_g = 90\\
\rowcolor{YellowGreen}  j = 90\\
\rowcolor{YellowGreen}  j\_g = 90\\
\rowcolor{YellowGreen}  k = 50\\
\rowcolor{YellowGreen}  k\_u = 50\\
\rowcolor{YellowGreen}  k\_l = 50\\
\rowcolor{YellowGreen}  k\_p1 = 51\\
\rowcolor{YellowGreen}  tile = 13\\
\rowcolor{YellowGreen}  nb = 4\\
\rowcolor{YellowGreen}  nv = 2\\
\hline

coordinates\\
\hline
\rowcolor{Apricot}\hspace{0.5cm}int32 i (i)\\
\rowcolor{Apricot}\hspace{0.5cm}\hspace{0.5cm}i:axis = "X"\\
\rowcolor{Apricot}\hspace{0.5cm}\hspace{0.5cm}i:long\_name = "grid index in x for variables at tracer and 'v' locations"\\
\rowcolor{Apricot}\hspace{0.5cm}\hspace{0.5cm}i:swap\_dim = "XC"\\
\rowcolor{Apricot}\hspace{0.5cm}\hspace{0.5cm}i:comment = "In the Arakawa C-grid system, tracer (e.g., THETA) and 'v' variables (e.g., VVEL) have the same x coordinate on the model grid."\\
\rowcolor{Apricot}\hspace{0.5cm}\hspace{0.5cm}i:coverage\_content\_type = "coordinate"\\
\rowcolor{Apricot}\hspace{0.5cm}int32 i\_g (i\_g)\\
\rowcolor{Apricot}\hspace{0.5cm}\hspace{0.5cm}i\_g:axis = "X"\\
\rowcolor{Apricot}\hspace{0.5cm}\hspace{0.5cm}i\_g:long\_name = "grid index in x for variables at 'u' and 'g' locations"\\
\rowcolor{Apricot}\hspace{0.5cm}\hspace{0.5cm}i\_g:c\_grid\_axis\_shift = "-0.5"\\
\rowcolor{Apricot}\hspace{0.5cm}\hspace{0.5cm}i\_g:swap\_dim = "XG"\\
\rowcolor{Apricot}\hspace{0.5cm}\hspace{0.5cm}i\_g:comment = "In the Arakawa C-grid system, 'u' (e.g., UVEL) and 'g' variables (e.g., XG) have the same x coordinate on the model grid."\\
\rowcolor{Apricot}\hspace{0.5cm}\hspace{0.5cm}i\_g:coverage\_content\_type = "coordinate"\\
\rowcolor{Apricot}\hspace{0.5cm}int32 j (j)\\
\rowcolor{Apricot}\hspace{0.5cm}\hspace{0.5cm}j:axis = "Y"\\
\rowcolor{Apricot}\hspace{0.5cm}\hspace{0.5cm}j:long\_name = "grid index in y for variables at tracer and 'u' locations"\\
\rowcolor{Apricot}\hspace{0.5cm}\hspace{0.5cm}j:swap\_dim = "YC"\\
\rowcolor{Apricot}\hspace{0.5cm}\hspace{0.5cm}j:comment = "In the Arakawa C-grid system, tracer (e.g., THETA) and 'u' variables (e.g., UVEL) have the same y coordinate on the model grid."\\
\rowcolor{Apricot}\hspace{0.5cm}\hspace{0.5cm}j:coverage\_content\_type = "coordinate"\\
\rowcolor{Apricot}\hspace{0.5cm}int32 j\_g (j\_g)\\
\rowcolor{Apricot}\hspace{0.5cm}\hspace{0.5cm}j\_g:axis = "Y"\\
\rowcolor{Apricot}\hspace{0.5cm}\hspace{0.5cm}j\_g:long\_name = "grid index in y for variables at 'v' and 'g' locations"\\
\rowcolor{Apricot}\hspace{0.5cm}\hspace{0.5cm}j\_g:c\_grid\_axis\_shift = "-0.5"\\
\rowcolor{Apricot}\hspace{0.5cm}\hspace{0.5cm}j\_g:swap\_dim = "YG"\\
\rowcolor{Apricot}\hspace{0.5cm}\hspace{0.5cm}j\_g:comment = "In the Arakawa C-grid system, 'v' (e.g., VVEL) and 'g' variables (e.g., XG) have the same y coordinate."\\
\rowcolor{Apricot}\hspace{0.5cm}\hspace{0.5cm}j\_g:coverage\_content\_type = "coordinate"\\
\rowcolor{Apricot}\hspace{0.5cm}int32 k (k)\\
\rowcolor{Apricot}\hspace{0.5cm}\hspace{0.5cm}k:axis = "Z"\\
\rowcolor{Apricot}\hspace{0.5cm}\hspace{0.5cm}k:long\_name = "grid index in z for tracer variables"\\
\rowcolor{Apricot}\hspace{0.5cm}\hspace{0.5cm}k:swap\_dim = "Z"\\
\rowcolor{Apricot}\hspace{0.5cm}\hspace{0.5cm}k:coverage\_content\_type = "coordinate"\\
\rowcolor{Apricot}\hspace{0.5cm}int32 k\_u (k\_u)\\
\rowcolor{Apricot}\hspace{0.5cm}\hspace{0.5cm}k\_u:axis = "Z"\\
\rowcolor{Apricot}\hspace{0.5cm}\hspace{0.5cm}k\_u:long\_name = "grid index in z corresponding to the bottom face of tracer grid cells ('w' locations)"\\
\rowcolor{Apricot}\hspace{0.5cm}\hspace{0.5cm}k\_u:c\_grid\_axis\_shift = "0.5"\\
\rowcolor{Apricot}\hspace{0.5cm}\hspace{0.5cm}k\_u:swap\_dim = "Zu"\\
\rowcolor{Apricot}\hspace{0.5cm}\hspace{0.5cm}k\_u:comment = "First index corresponds to the bottom face of the uppermost tracer grid cell. The use of 'u' in the variable name follows the MITgcm convention for naming the bottom face of ocean tracer grid cells."\\
\rowcolor{Apricot}\hspace{0.5cm}\hspace{0.5cm}k\_u:coverage\_content\_type = "coordinate"\\
\rowcolor{Apricot}\hspace{0.5cm}int32 k\_l (k\_l)\\
\rowcolor{Apricot}\hspace{0.5cm}\hspace{0.5cm}k\_l:axis = "Z"\\
\rowcolor{Apricot}\hspace{0.5cm}\hspace{0.5cm}k\_l:long\_name = "grid index in z corresponding to the top face of tracer grid cells ('w' locations)"\\
\rowcolor{Apricot}\hspace{0.5cm}\hspace{0.5cm}k\_l:c\_grid\_axis\_shift = "-0.5"\\
\rowcolor{Apricot}\hspace{0.5cm}\hspace{0.5cm}k\_l:swap\_dim = "Zl"\\
\rowcolor{Apricot}\hspace{0.5cm}\hspace{0.5cm}k\_l:comment = "First index corresponds to the top face of the uppermost tracer grid cell. The use of 'l' in the variable name follows the MITgcm convention for naming the top face of ocean tracer grid cells."\\
\rowcolor{Apricot}\hspace{0.5cm}\hspace{0.5cm}k\_l:coverage\_content\_type = "coordinate"\\
\rowcolor{Apricot}\hspace{0.5cm}int32 k\_p1 (k\_p1)\\
\rowcolor{Apricot}\hspace{0.5cm}\hspace{0.5cm}k\_p1:axis = "Z"\\
\rowcolor{Apricot}\hspace{0.5cm}\hspace{0.5cm}k\_p1:long\_name = "grid index in z for variables at 'w' locations"\\
\rowcolor{Apricot}\hspace{0.5cm}\hspace{0.5cm}k\_p1:c\_grid\_axis\_shift = "[-0.5  0.5]"\\
\rowcolor{Apricot}\hspace{0.5cm}\hspace{0.5cm}k\_p1:swap\_dim = "Zp1"\\
\rowcolor{Apricot}\hspace{0.5cm}\hspace{0.5cm}k\_p1:comment = "Includes top of uppermost model tracer cell (k\_p1=0) and bottom of lowermost tracer cell (k\_p1=51)."\\
\rowcolor{Apricot}\hspace{0.5cm}\hspace{0.5cm}k\_p1:coverage\_content\_type = "coordinate"\\
\rowcolor{Apricot}\hspace{0.5cm}int32 tile (tile)\\
\rowcolor{Apricot}\hspace{0.5cm}\hspace{0.5cm}tile:long\_name = "lat-lon-cap tile index"\\
\rowcolor{Apricot}\hspace{0.5cm}\hspace{0.5cm}tile:comment = "The ECCO V4 horizontal model grid is divided into 13 tiles of 90x90 cells for convenience."\\
\rowcolor{Apricot}\hspace{0.5cm}\hspace{0.5cm}tile:coverage\_content\_type = "coordinate"\\
\rowcolor{Apricot}\hspace{0.5cm}float32 XC (tile, j, i)\\
\rowcolor{Apricot}\hspace{0.5cm}\hspace{0.5cm}XC:long\_name = "longitude of tracer grid cell center"\\
\rowcolor{Apricot}\hspace{0.5cm}\hspace{0.5cm}XC:units = "degrees\_east"\\
\rowcolor{Apricot}\hspace{0.5cm}\hspace{0.5cm}XC:coordinate = "YC XC"\\
\rowcolor{Apricot}\hspace{0.5cm}\hspace{0.5cm}XC:bounds = "XC\_bnds"\\
\rowcolor{Apricot}\hspace{0.5cm}\hspace{0.5cm}XC:comment = "nonuniform grid spacing"\\
\rowcolor{Apricot}\hspace{0.5cm}\hspace{0.5cm}XC:coverage\_content\_type = "coordinate"\\
\rowcolor{Apricot}\hspace{0.5cm}\hspace{0.5cm}XC:standard\_name = "longitude"\\
\rowcolor{Apricot}\hspace{0.5cm}float32 YC (tile, j, i)\\
\rowcolor{Apricot}\hspace{0.5cm}\hspace{0.5cm}YC:long\_name = "latitude of tracer grid cell center"\\
\rowcolor{Apricot}\hspace{0.5cm}\hspace{0.5cm}YC:units = "degrees\_north"\\
\rowcolor{Apricot}\hspace{0.5cm}\hspace{0.5cm}YC:coordinate = "YC XC"\\
\rowcolor{Apricot}\hspace{0.5cm}\hspace{0.5cm}YC:bounds = "YC\_bnds"\\
\rowcolor{Apricot}\hspace{0.5cm}\hspace{0.5cm}YC:comment = "nonuniform grid spacing"\\
\rowcolor{Apricot}\hspace{0.5cm}\hspace{0.5cm}YC:coverage\_content\_type = "coordinate"\\
\rowcolor{Apricot}\hspace{0.5cm}\hspace{0.5cm}YC:standard\_name = "latitude"\\
\rowcolor{Apricot}\hspace{0.5cm}float32 XG (tile, j\_g, i\_g)\\
\rowcolor{Apricot}\hspace{0.5cm}\hspace{0.5cm}XG:long\_name = "longitude of 'southwest' corner of tracer grid cell"\\
\rowcolor{Apricot}\hspace{0.5cm}\hspace{0.5cm}XG:units = "degrees\_east"\\
\rowcolor{Apricot}\hspace{0.5cm}\hspace{0.5cm}XG:coordinate = "YG XG"\\
\rowcolor{Apricot}\hspace{0.5cm}\hspace{0.5cm}XG:comment = "Nonuniform grid spacing. Note: 'southwest' does not correspond to geographic orientation but is used for convenience to describe the computational grid. See MITgcm dcoumentation for details."\\
\rowcolor{Apricot}\hspace{0.5cm}\hspace{0.5cm}XG:coverage\_content\_type = "coordinate"\\
\rowcolor{Apricot}\hspace{0.5cm}\hspace{0.5cm}XG:standard\_name = "longitude"\\
\rowcolor{Apricot}\hspace{0.5cm}float32 YG (tile, j\_g, i\_g)\\
\rowcolor{Apricot}\hspace{0.5cm}\hspace{0.5cm}YG:long\_name = "latitude of 'southwest' corner of tracer grid cell"\\
\rowcolor{Apricot}\hspace{0.5cm}\hspace{0.5cm}YG:units = "degrees\_north"\\
\rowcolor{Apricot}\hspace{0.5cm}\hspace{0.5cm}YG:comment = "Nonuniform grid spacing. Note: 'southwest' does not correspond to geographic orientation but is used for convenience to describe the computational grid. See MITgcm dcoumentation for details."\\
\rowcolor{Apricot}\hspace{0.5cm}\hspace{0.5cm}YG:coverage\_content\_type = "coordinate"\\
\rowcolor{Apricot}\hspace{0.5cm}\hspace{0.5cm}YG:standard\_name = "latitude"\\
\rowcolor{Apricot}\hspace{0.5cm}\hspace{0.5cm}YG:coordinates = "YG XG"\\
\rowcolor{Apricot}\hspace{0.5cm}float32 Z (k)\\
\rowcolor{Apricot}\hspace{0.5cm}\hspace{0.5cm}Z:long\_name = "depth of tracer grid cell center"\\
\rowcolor{Apricot}\hspace{0.5cm}\hspace{0.5cm}Z:units = "m"\\
\rowcolor{Apricot}\hspace{0.5cm}\hspace{0.5cm}Z:positive = "up"\\
\rowcolor{Apricot}\hspace{0.5cm}\hspace{0.5cm}Z:bounds = "Z\_bnds"\\
\rowcolor{Apricot}\hspace{0.5cm}\hspace{0.5cm}Z:comment = "Non-uniform vertical spacing."\\
\rowcolor{Apricot}\hspace{0.5cm}\hspace{0.5cm}Z:coverage\_content\_type = "coordinate"\\
\rowcolor{Apricot}\hspace{0.5cm}\hspace{0.5cm}Z:standard\_name = "depth"\\
\rowcolor{Apricot}\hspace{0.5cm}float32 Zp1 (k\_p1)\\
\rowcolor{Apricot}\hspace{0.5cm}\hspace{0.5cm}Zp1:long\_name = "depth of top/bottom face of tracer grid cell"\\
\rowcolor{Apricot}\hspace{0.5cm}\hspace{0.5cm}Zp1:units = "m"\\
\rowcolor{Apricot}\hspace{0.5cm}\hspace{0.5cm}Zp1:positive = "up"\\
\rowcolor{Apricot}\hspace{0.5cm}\hspace{0.5cm}Zp1:comment = "Contains one element more than the number of vertical layers. First element is 0m, the depth of the top face of the uppermost grid cell. Last element is the depth of the bottom face of the deepest grid cell."\\
\rowcolor{Apricot}\hspace{0.5cm}\hspace{0.5cm}Zp1:coverage\_content\_type = "coordinate"\\
\rowcolor{Apricot}\hspace{0.5cm}\hspace{0.5cm}Zp1:standard\_name = "depth"\\
\rowcolor{Apricot}\hspace{0.5cm}float32 Zu (k\_u)\\
\rowcolor{Apricot}\hspace{0.5cm}\hspace{0.5cm}Zu:long\_name = "depth of bottom face of tracer grid cell"\\
\rowcolor{Apricot}\hspace{0.5cm}\hspace{0.5cm}Zu:units = "m"\\
\rowcolor{Apricot}\hspace{0.5cm}\hspace{0.5cm}Zu:positive = "up"\\
\rowcolor{Apricot}\hspace{0.5cm}\hspace{0.5cm}Zu:comment = "First element is -10m, the depth of the bottom face of the uppermost tracer grid cell. Last element is the depth of the bottom face of the deepest grid cell. The use of 'u' in the variable name follows the MITgcm convention for naming the bottom face of ocean tracer grid cells."\\
\rowcolor{Apricot}\hspace{0.5cm}\hspace{0.5cm}Zu:coverage\_content\_type = "coordinate"\\
\rowcolor{Apricot}\hspace{0.5cm}\hspace{0.5cm}Zu:standard\_name = "depth"\\
\rowcolor{Apricot}\hspace{0.5cm}float32 Zl (k\_l)\\
\rowcolor{Apricot}\hspace{0.5cm}\hspace{0.5cm}Zl:long\_name = "depth of top face of tracer grid cell"\\
\rowcolor{Apricot}\hspace{0.5cm}\hspace{0.5cm}Zl:units = "m"\\
\rowcolor{Apricot}\hspace{0.5cm}\hspace{0.5cm}Zl:positive = "up"\\
\rowcolor{Apricot}\hspace{0.5cm}\hspace{0.5cm}Zl:comment = "First element is 0m, the depth of the top face of the uppermost tracer grid cell (i.e., the ocean surface). Last element is the depth of the top face of the deepest grid cell. The use of 'l' in the variable name follows the MITgcm convention for naming the top face of ocean tracer grid cells."\\
\rowcolor{Apricot}\hspace{0.5cm}\hspace{0.5cm}Zl:coverage\_content\_type = "coordinate"\\
\rowcolor{Apricot}\hspace{0.5cm}\hspace{0.5cm}Zl:standard\_name = "depth"\\
\rowcolor{Apricot}\hspace{0.5cm}float32 XC\_bnds (tile, j, i, nb)\\
\rowcolor{Apricot}\hspace{0.5cm}\hspace{0.5cm}XC\_bnds:comment = "Bounds array follows CF conventions. XC\_bnds[i,j,0] = 'southwest' corner (j-1, i-1), XC\_bnds[i,j,1] = 'southeast' corner (j-1, i+1), XC\_bnds[i,j,2] = 'northeast' corner (j+1, i+1), XC\_bnds[i,j,3]  = 'northwest' corner (j+1, i-1). Note: 'southwest', 'southeast', northwest', and 'northeast' do not correspond to geographic orientation but are used for convenience to describe the computational grid. See MITgcm dcoumentation for details."\\
\rowcolor{Apricot}\hspace{0.5cm}\hspace{0.5cm}XC\_bnds:coverage\_content\_type = "coordinate"\\
\rowcolor{Apricot}\hspace{0.5cm}\hspace{0.5cm}XC\_bnds:long\_name = "longitudes of tracer grid cell corners"\\
\rowcolor{Apricot}\hspace{0.5cm}float32 YC\_bnds (tile, j, i, nb)\\
\rowcolor{Apricot}\hspace{0.5cm}\hspace{0.5cm}YC\_bnds:comment = "Bounds array follows CF conventions. YC\_bnds[i,j,0] = 'southwest' corner (j-1, i-1), YC\_bnds[i,j,1] = 'southeast' corner (j-1, i+1), YC\_bnds[i,j,2] = 'northeast' corner (j+1, i+1), YC\_bnds[i,j,3]  = 'northwest' corner (j+1, i-1). Note: 'southwest', 'southeast', northwest', and 'northeast' do not correspond to geographic orientation but are used for convenience to describe the computational grid. See MITgcm dcoumentation for details."\\
\rowcolor{Apricot}\hspace{0.5cm}\hspace{0.5cm}YC\_bnds:coverage\_content\_type = "coordinate"\\
\rowcolor{Apricot}\hspace{0.5cm}\hspace{0.5cm}YC\_bnds:long\_name = "latitudes of tracer grid cell corners"\\
\rowcolor{Apricot}\hspace{0.5cm}float32 Z\_bnds (k, nv)\\
\rowcolor{Apricot}\hspace{0.5cm}\hspace{0.5cm}Z\_bnds:comment = "One pair of depths for each vertical level."\\
\rowcolor{Apricot}\hspace{0.5cm}\hspace{0.5cm}Z\_bnds:coverage\_content\_type = "coordinate"\\
\rowcolor{Apricot}\hspace{0.5cm}\hspace{0.5cm}Z\_bnds:long\_name = "depths of top and bottom faces of tracer grid cell"\\
\hline

data variables\\
\hline
\hspace{0.5cm}float32 DIFFKR (k, tile, j, i)\\
\hspace{0.5cm}\hspace{0.5cm}DIFFKR:\_FillValue = "9.969209968386869e+36"\\
\hspace{0.5cm}\hspace{0.5cm}DIFFKR:coverage\_content\_type = "modelResult"\\
\hspace{0.5cm}\hspace{0.5cm}DIFFKR:long\_name = "Vertical diffusivity"\\
\hspace{0.5cm}\hspace{0.5cm}DIFFKR:units = "m2 s-1"\\
\hspace{0.5cm}\hspace{0.5cm}DIFFKR:comment = "Background vertical diffusion coefficient for temperature and salinity. Total vertical diffusivity includes background diffusivity plus contributions from the GGL90 vertical mixing and the Gent-McWilliams/Redi parameterizations. Note: DIFFKR is a model control variable and has been optimized from a spatially-invariant first-guess value of 1e-5 m2 s-1."\\
\hspace{0.5cm}\hspace{0.5cm}DIFFKR:valid\_min = "9.999999974752427e-07"\\
\hspace{0.5cm}\hspace{0.5cm}DIFFKR:valid\_max = "0.00018549950618762523"\\
\hspace{0.5cm}\hspace{0.5cm}DIFFKR:coordinates = "Z XC YC"\\
\hspace{0.5cm}float32 KAPGM (k, tile, j, i)\\
\hspace{0.5cm}\hspace{0.5cm}KAPGM:\_FillValue = "9.969209968386869e+36"\\
\hspace{0.5cm}\hspace{0.5cm}KAPGM:coverage\_content\_type = "modelResult"\\
\hspace{0.5cm}\hspace{0.5cm}KAPGM:long\_name = "Gent-McWilliams diffusivity"\\
\hspace{0.5cm}\hspace{0.5cm}KAPGM:units = "m2 s-1"\\
\hspace{0.5cm}\hspace{0.5cm}KAPGM:comment = "Gent-McWilliams diffusivity coefficient as described in Gent and McWilliams (1990, JPO). Note: KAPGM is a model control variable and has been optimized from a spatially invariant first guess of 1e3 m2 s-1."\\
\hspace{0.5cm}\hspace{0.5cm}KAPGM:valid\_min = "100.0"\\
\hspace{0.5cm}\hspace{0.5cm}KAPGM:valid\_max = "10000.0"\\
\hspace{0.5cm}\hspace{0.5cm}KAPGM:coordinates = "Z XC YC"\\
\hspace{0.5cm}float32 KAPREDI (k, tile, j, i)\\
\hspace{0.5cm}\hspace{0.5cm}KAPREDI:\_FillValue = "9.969209968386869e+36"\\
\hspace{0.5cm}\hspace{0.5cm}KAPREDI:coverage\_content\_type = "modelResult"\\
\hspace{0.5cm}\hspace{0.5cm}KAPREDI:long\_name = "Along-isopycnal diffusivity"\\
\hspace{0.5cm}\hspace{0.5cm}KAPREDI:units = "m2 s-1"\\
\hspace{0.5cm}\hspace{0.5cm}KAPREDI:comment = "Redi along-isopycnal diffusivity coefficient as described in Redi (1982, JPO). Note: KAPREDI is a model control variable and has been optimized from a spatially invariant first guess of 1e3 m2 s-1."\\
\hspace{0.5cm}\hspace{0.5cm}KAPREDI:valid\_min = "100.0"\\
\hspace{0.5cm}\hspace{0.5cm}KAPREDI:valid\_max = "10000.0"\\
\hspace{0.5cm}\hspace{0.5cm}KAPREDI:coordinates = "Z XC YC"\\
\hline
\end{longtable} % inserting contents from a python generated tex file
\pagebreak
\subsection{Example of ECCO V4r4 netCDF for 1D data}
\begin{longtable}{|p{\textwidth}|}
\caption{Example CDL description of 1D dataset}
\label{tab:cdl-1D} \\
\hline \endhead
\hline \endfoot
netcdf 1D example\\
dimensions\\
\hline
\rowcolor{YellowGreen}  time = 227904\\
\hline

coordinates\\
\hline
\rowcolor{Apricot}\hspace{0.5cm}int32 time (time)\\
\rowcolor{Apricot}\hspace{0.5cm}\hspace{0.5cm}time:axis = "T"\\
\rowcolor{Apricot}\hspace{0.5cm}\hspace{0.5cm}time:comment = ""\\
\rowcolor{Apricot}\hspace{0.5cm}\hspace{0.5cm}time:coverage\_content\_type = "coordinate"\\
\rowcolor{Apricot}\hspace{0.5cm}\hspace{0.5cm}time:long\_name = "snapshot time"\\
\rowcolor{Apricot}\hspace{0.5cm}\hspace{0.5cm}time:standard\_name = "time"\\
\rowcolor{Apricot}\hspace{0.5cm}\hspace{0.5cm}time:units = "hours since 1992-01-01T12:00:00"\\
\rowcolor{Apricot}\hspace{0.5cm}\hspace{0.5cm}time:calendar = "proleptic\_gregorian"\\
\hline

data variables\\
\hline
\hspace{0.5cm}float64 Pa\_global (time)\\
\hspace{0.5cm}\hspace{0.5cm}Pa\_global:\_FillValue = "9.969209968386869e+36"\\
\hspace{0.5cm}\hspace{0.5cm}Pa\_global:coverage\_content\_type = "modelResult"\\
\hspace{0.5cm}\hspace{0.5cm}Pa\_global:long\_name = "Global mean atmospheric surface pressure over the ocean and sea-ice"\\
\hspace{0.5cm}\hspace{0.5cm}Pa\_global:standard\_name = "air\_pressure\_at\_sea\_level"\\
\hspace{0.5cm}\hspace{0.5cm}Pa\_global:units = "N m-2"\\
\hspace{0.5cm}\hspace{0.5cm}Pa\_global:valid\_min = "100873.14755283327"\\
\hspace{0.5cm}\hspace{0.5cm}Pa\_global:valid\_max = "101257.45252296235"\\
\hspace{0.5cm}\hspace{0.5cm}Pa\_global:coordinates = "time"\\
\hline
\end{longtable}

